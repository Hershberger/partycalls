\documentclass[12pt]{article}
\usepackage{setspace, graphicx, fullpage, amssymb, amsmath, epsfig, natbib, array, multirow}
\usepackage[bookmarks = false, hidelinks]{hyperref}
\usepackage{amsfonts, bm}
\usepackage{dcolumn}
\usepackage{subfigure, float}
\usepackage[margin=1in]{geometry}
\usepackage{verbatim}
\usepackage{url}
\usepackage{enumerate}
\usepackage{morefloats}
\usepackage[skip = 2pt]{caption}
\usepackage[flushleft]{threeparttable}
\usepackage[T1]{fontenc}
\usepackage{libertine}
\usepackage{scrextend}
\usepackage{setspace}
\usepackage{fancyhdr}
\usepackage{enumitem}
\usepackage{natbib}
\usepackage{multibib}

\newcommand\fnote[1]{\captionsetup{font=normalsize}\caption*{#1}}

\setlist{nolistsep,noitemsep}
\setlength{\footnotesep}{1.2\baselineskip}
\deffootnote[.2in]{0in}{0em}{\normalsize\thefootnotemark.\enskip}
\newcolumntype{.}{D{.}{.}{-1}}
\newcolumntype{d}[1]{D{.}{.}{#1}}

\makeatother

\def\citeapos#1{\citeauthor{#1}'s (\citeyear{#1})} % possessive citation

\begin{document}

\bibliographystyle{apsr}

\title{Party Calls and Reelection in the U.S. Senate\thanks{
We would like to thank Andrew Podob for reviewing a draft of this paper.
ADD FURTHER ACKNOWLEDGEMENTS?}
}

\author{
  Ethan Hershberger\thanks{
    \small Ohio State University
  }\quad
  William Minozzi$^\dagger$\quad
  Craig Volden\thanks{
    \small University of Virginia
  }\\
}

\date{\today}

\maketitle

\begin{center}

\vspace{.1in}
Word Count: XXXX
\end{center}
\vspace{.25in}

\begin{abstract}
\singlespacing
\noindent
\cite{Minozzi:2013} advance the idea that a substantial portion of partisan voting activity in Congress is a simple call to unity that is especially easily embraced by ideological extremists.  If correct, their findings should extend from the House to the Senate, despite Senate leaders lacking many of the sticks and carrots found in the House.  We adapt the theory and measurement of party calls to the Senate.  In so doing, we find that both the House and the Senate have relied heavily (and increasingly) on party calls over the past four decades.  In the Senate in particular, the lens of party calls opens new opportunities for scholars to explore partisan legislative behavior.  We take advantage of one such opportunity to show how electoral concerns limit Senators' responsiveness to party calls.
\end{abstract}

\clearpage
\setcounter{page}{1}

\doublespacing

\noindent When scholars think of partisan influence in legislative voting they often conceive of pressure exerted on fence-sitting legislators to win them over to the party leaders' preferred position.  Given such a conception, party leaders in the Senate are thought to be less influential than their House counterparts because they possess fewer tools and opportunities to exert pressure.\footnote{\doublespacing\normalsize Yet, recent work has shown significant party effects in the Senate \citep[e.g.,][]{Gailmard:2007, Monroe:2008, Patty:2008, Volden:2006}.}  In contrast with a pressure-based approach to understanding parties, \cite{Minozzi:2013} offer a theory of party calls.  They suggest that, on many issues, unity among party members serves broad partisan ends, such as brand development \citep[e.g.,][]{Snyder:2002}, apart from attempts to win closely contested votes on legislation. As a result, leaders' attempts to call party members to vote together will be more influential on extremists (who benefit from a coherent brand) than on cross-pressured moderates. Minozzi and Volden (MV) identify party-call votes and find enhanced responsiveness to them among extremists in the U.S. House.

We argue that the logic of party calls extends to the U.S. Senate, even in the absence of strong tools to exert control over the institution's individualistic members. We adapt MV's approach to the Senate, and find strong evidence of party calls uniting party members, especially inducing ideological extremists to vote with the rest of the party, above and beyond their natural tendencies.

In addition to extending the examination of party calls to the Senate, we expand the time period of examination from MV to include the 93$^{\text{rd}}$ through the 112$^{\text{th}}$ Congresses (1973-2012) for both the House and the Senate. Consistent with other evidence of partisan polarization across this era, we see a steady rise in the use of party calls over time across both chambers and regardless of Democrat or Republican control. Finally, we argue that extending the study of party calls to the Senate opens numerous opportunities for important new scholarship. Illustrating one such possibility, we show that Senators up for reelection are significantly less responsive to party calls, consistent with bucking the party when it is out of line with voters in their home states.

\subsection*{Party Calls in the Senate}

To explore party calls, MV use a two-step process, first dividing floor votes into ``party-free votes" and ``party calls," and then examining which legislators are most responsive to party calls beyond their baseline partisan support on party-free votes. Identifying party calls is itself a tricky process, requiring researchers to identify a strong party-based voting pattern apart from that produced by ideological differences alone. We mimic the iterative approach in MV that produces the list of party-call votes and simultaneously generates party-free ideal points for each legislator.\footnote{\doublespacing\normalsize As detailed in the Supplemental Appendix A, minor modifications were made to adapt this method to the Senate and to reduce computational burdens.}

As detailed in the Supplemental Appendix, the party calls identified through this process broadly reflect properties expected of partisan votes. For example, party calls tend to drive a further wedge between Democrats and Republicans who are already initially separated on ideological grounds. Moreover, party calls are more likely to occur among close votes than lopsided votes, perhaps because the close votes offer greater opportunity for parties to differentiate themselves.  In total, these patterns lend confidence that our classification of party-call votes is meaningful in capturing broad and consistent underlying behavior.

Figure~\ref{fig-party-calls-over-time} illustrates the percentage of party calls among all votes in the House and Senate. Across both chambers, the use of party calls has been steadily increasing over time.\footnote{\doublespacing\normalsize Because of the different threshold for identifying party calls across chambers, as well as their different membership sizes, direct House-Senate comparisons in the proportion of party-call votes would be inappropriate.} It is possible that parties are calling more because partisan ideological alignment has made party calls more effective, or because parties need members to rally more to get their agendas through a heavily gridlocked Congress. We believe that the underlying cause of this trend merits further investigation, but initially take it as evidence of much discussed increased partisanship and polarization in recent decades (e.g.,  \cite{Aldrich:2000, Lee:2009, Lee:2016, Theriault:2013, Smith:2014}). This trend holds regardless of which party is in the majority in the chamber.

\begin{figure}[p]
\centering
\includegraphics[width = \textwidth]{party_call_percent_both.pdf}
\caption{Party calls as a percentage of all votes, 1973-2012.
This figure shows the percentage of votes classified as party calls per Congress in each chamber. Blue circles denote Democrat-majority Congresses while Red triangles denote Republican-majority Congresses.
\label{fig-party-calls-over-time}}
\end{figure}

Fundamental to the theory of party calls is the idea that substantial partisan activity in Congress is designed to align members who might otherwise wander away from the party for idiosyncratic reasons, instead of solely winning close votes. As such, those most responsive to party calls are the extremists who tend to benefit from coherent party positions, rather than cross-pressured, fence-sitting moderates. To test this ``responsive extremists hypothesis," we run a series of OLS regressions, with a dependent variable measuring each legislator's percent support for the party position on party-call votes. We cluster the observations by legislator to account for possible dependence over time. Table~\ref{tab-responsiveness-regressions} shows that the responsive extremists hypothesis holds up well in the extended period over which we examine the House, and nearly as prominently in the Senate.

\begin{table}[!htbp]
\centering
\begin{threeparttable}
\caption{Models of Responsiveness to Party Calls, 1973-2012}
\label{tab-responsiveness-regressions}
\singlespacing
\begin{tabular}{l c c c }
\hline
& House & \multicolumn{2}{c}{Senate} \\
\hline

Ideological Extremism & $7.75^{***}$ & $6.29^{***}$ & $6.24^{***}$  \\
                      & $(1.26)$     & $(0.83)$     & $(0.83)$      \\
Baseline Rate of Voting with Party&$0.57^{***}$&$0.73^{***}$&$0.74^{***}$\\
                      & $(0.12)$     & $(0.07)$     & $(0.07)$      \\
Up For Reelection     &              &              & $-0.94^{***}$ \\
                      &              &              & $(0.19)$      \\
Vote Share            & $-0.01$      & $0.03$       & $0.03$        \\
                      & $(0.03)$     & $(0.03)$     & $(0.03)$      \\
Pres Vote Share       & $0.03$       & $0.10$       & $0.10$        \\
                      & $(0.08)$     & $(0.05)$     & $(0.05)$      \\
Leader                & $1.81^{**}$  & $1.65^{*}$   & $1.64^{*}$    \\
                      & $(0.56)$     & $(0.72)$     & $(0.72)$      \\
Chair                 & $4.98^{***}$ & $2.09^{**}$  & $2.06^{**}$   \\
                      & $(0.95)$     & $(0.79)$     & $(0.79)$      \\
Power Committee       & $2.76^{***}$ & $-0.66$      & $-0.66$       \\
                      & $(0.76)$     & $(0.72)$     & $(0.72)$      \\
Best Committee        & $-0.17$      & $0.15$       & $0.15$        \\
                      & $(0.10)$     & $(0.14)$     & $(0.14)$      \\
Female                & $1.17$       & $2.05^{*}$   & $2.03^{*}$    \\
                      & $(0.63)$     & $(0.88)$     & $(0.88)$      \\
African American      & $1.90$       & $-4.56$      & $-4.67$       \\
                      & $(1.36)$     & $(2.48)$     & $(2.45)$      \\
Latino                & $3.27^{**}$  & $5.60^{*}$   & $5.66^{*}$    \\
                      & $(1.16)$     & $(2.54)$     & $(2.52)$      \\
South                 & $-0.89$      & $0.62$       & $0.61$        \\
                      & $(0.54)$     & $(0.70)$     & $(0.70)$      \\
Seniority             & $-0.05$      & $0.01$       & $0.01$        \\
                      & $(0.06)$     & $(0.07)$     & $(0.07)$      \\
Freshman              & $0.78$       & $1.08^{*}$   & $0.81$        \\
                      & $(0.67)$     & $(0.54)$     & $(0.56)$      \\
Intercept             & $31.78^{**}$ & $11.71$      & $11.94$       \\
                      & $(11.77)$    & $(6.91)$     & $(6.88)$      \\
\hline
R$^2$                 & 0.46         & 0.63         & 0.63          \\
Adj. R$^2$            & 0.46         & 0.63         & 0.63          \\
Num. obs.             & 8544         & 1993         & 1993          \\
RMSE                  & 8.44         & 6.99         & 6.98          \\
\hline

\end{tabular}
\begin{tablenotes}
   \item
   The table presents linear models of \textit{Responsiveness to Party Calls},
   from the 93$^{\text{rd}}$-112$^{\text{th}}$ Congresses (1973-2012).
  Standard errors are clustered by Congress and by member
   $^{***}p<0.001$, $^{**}p<0.01$, $^*p<0.05$
 \end{tablenotes}
\end{threeparttable}
\end{table}

We include all of the control variables used by MV, thus accounting for the \textit{Baseline Rate of Voting with Party} on the party-free votes.\footnote{\doublespacing\normalsize Descriptions and summary statistics for all control variables are given in the Supplemental Appendix C.} Although some House-Senate differences emerge, such as for Southerners or those on power committees, many broad patterns are consistent across institutions.\footnote{\doublespacing\normalsize Given the small size of the Senate, most members are included in the top committees, likely limiting the variance that allowed patterns to be discerned in the House based on committee assignments.} For example, positive coefficients show \textit{Party Leaders} and \textit{Committee Chairs} to be understandably highly responsive to party calls.

As shown by the coefficients on the \textit{Ideological Extremism} variable, and in strong support of the theory of party calls, each one-standard-deviation increase in ideological extremism is associated with a nearly eight-percentage-point increase in voting with their party in the House, and more than a six-percentage-point increase in the Senate.\footnote{\doublespacing\normalsize These effect sizes are consistent with those uncovered by MV in the House between 1973 and 2006.} This party alignment extends above and beyond the baseline level of support of the member's party on non-party-call votes.

As detailed in the Supplemental Appendix, the pattern of ideological extremists being more responsive to party calls than moderates holds for both Democrats and Republicans, for both those in the majority party and those in the minority party. This robust support is illustrated in Figure~\ref{fig-extremism-responsiveness} based on separate regressions for each party in each Congress, a very tough test of the responsive extremists hypothesis, especially given the small membership of the Senate. In the House, ideological extremism takes a positive coefficient for all but four cases, and in the Senate for all but two. Further, in the House, all positive coefficients are statistically significant; in the Senate, 29 of the 38 positive coefficients are statistically significant, while only one negative coefficient is. These results add confidence to the aggregate findings above.

\begin{figure}[!htbp]
\centering
\includegraphics{both-chambers-figure2.pdf}
\caption{Extremists are responsive to party calls in both chambers.
This coefficient plot is produced by the same formula shown in the House and Senate regression table with results decomposed for the majority and minority parties in each of the 93$^{\text{rd}}$-112$^{\text{th}}$ Congresses (1973-2012). Coefficients shown are for the effect of ideological extremism on party-call votes, with both 50\% and 95\% confidence intervals shown.
\label{fig-extremism-responsiveness}}
\end{figure}

In sum, the results reported here offer a coherent portrait of party calls in both chambers of the U.S. Congress. Party-call votes are widespread and increasing over the past four decades. They divide Democrats from Republicans, but especially draw ideologically extreme party members toward a unified partisan position.

Beyond uncovering such a systematic and important partisan process in Congress, the party calls identified here offer the potential to significantly enhance scholarly exploration of parties in the Senate. New research based on party calls could contribute to a fuller understanding of partisan practices and norms spreading from the House to the Senate \citep[e.g.,][]{Theriault:2013}, of the role of partisanship in advancing and overcoming filibusters \citep[e.g.,][]{Wawro:2004}, and of electoral constraints on partisan behavior \citep[e.g.,][]{Levitt:1996}, to name a few opportunities.

To illustrate the usefulness of party calls, we here briefly tackle the last of these possibilities. Specifically, in the final column of Table~\ref{tab-responsiveness-regressions}, we include an indicator variable for whether a Senator is in her final Congress before reelection. We expect that Senators are more free to help develop a party brand, even in contrast to their constituents' preferences, when electoral threats are behind them than when the next election is imminent. Consistent with this hypothesis, Table~\ref{tab-responsiveness-regressions} shows about one percentage point lower responsiveness to party calls among those facing reelection than among those in their first four years following an election, all else equal. In the next section, we explore this result more fully.

\subsection*{Reelection Limits Responsiveness to Party Calls}

We argue that Senators up for reelection will be more attuned to their electoral needs than to broad party interests. Put simply, they will use some party-call votes to aid their personal -- rather than the party -- brand
\citep[e.g.,][]{Canes-Wrone:2002, Carson:2010}.  Such a pattern was initially detected above.

As an alternate test, we here rely on same-state Senator pairs in which one Senator is up for reelection at the end of the Congress. These pairings are ideal because same-state Senators are elected by the same constituents, but not at the same time. This allows us to estimate a generalized difference-in-differences design. Apart from reelection, any additional differences between members from the same state should cancel each other out across Senators and over time, limiting the possibility of spuriously significant findings. Based on the logic above, we expect that the Senator up for reelection will be less responsive to party calls than her same-state Senate partner. In contrast, we expect no difference between the baseline rate of voting with the party absent party calls.

\begin{figure}[!htbp]
\centering
\includegraphics[width = 12cm]{senate_difference_estimates.pdf}

\caption{Senators up for reelection are less responsive to party calls.
This coefficient plot is produced by a paired differences model that uses same-state Senators as a natural pairing. Estimates shown are the difference in responsiveness to party calls and the baseline rate of voting with the party. Results are from OLS models, with 50\% and 95\% confidence intervals shown.
\label{fig-reelection-responsiveness}}
\end{figure}

Figure~\ref{fig-reelection-responsiveness} shows that member responsiveness to party calls declines, on average, about 1.5\% when they are up for reelection.\footnote{\doublespacing\normalsize In an alternative approach, we compare all three possibilities of same-state Senator pairs.  Comparing those up for reelection to those in their first Congress after election shows a similar effect to that in Figure~\ref{fig-reelection-responsiveness}.  Likewise, comparing those up for reelection to those in the middle two years of their six-year term reveals less responsiveness to party calls for those up for reelection.  In contrast, comparing same-state Senators, neither of whom are up for reelection, we find no difference in responsiveness.  Across all three cases, we find no differences in their rate of voting with the party on non-party-call votes, which acts as a placebo test.} However, their baseline rate of voting with the party is unchanged by electoral considerations. During the period of analysis, the average rate of responsiveness to party calls is 85\%. Across the XXX party-call votes in the average Senate, a Senator not up for reelection therefore averages XXX defection from her party.  In contrast, those up for reelection defect from party calls XXX times on average, about a ten percent increase.  Such deviations may limit the party's effectiveness in lawmaking and in developing a coherent brand.  But they offer the election-seeking Senator an opportunity to build her support and reputation back home.

\subsection*{Conclusion}

In this paper, we established that legislators respond to party calls in the Senate as they do in the House. In both chambers, party calls have become more prominent over the past forty years.  In line with expectations, when leaders issue party calls, legislators align with their party, with the greatest effect being among ideological extremists.

One especially noteworthy finding is therefore the nature of the relationship we uncover between party and ideology.  In contrast to \citeapos{Krehbiel:1993} view that partisanship is often merely a reflection of ideology, or \citeapos{Lee:2009} view that party extends well beyond ideology, we find party and ideology to be largely complementary but distinct.  When the party calls its members together, those who benefit the most---typically ideological extremists---respond most vigorously to the call.

We further showed the value of party calls as a tool for studying legislative behavior.  We found that reelection reduced member responsiveness to party calls.  Under electoral threat, constituent preferences are front and center in Senators' minds, and thus we hypothesized---and found---that members up for reelection are less responsive to the party. This finding shows one of the limits to the party-building efforts inherent in party calls.

\clearpage

%------------------------------------------------------------------------------%
%\printbibliography[title={References}]
\bibliographystyle{apsr}
\bibliography{senate}
%------------------------------------------------------------------------------%

\end{document}
