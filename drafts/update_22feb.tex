\documentclass[12pt]{article}
\usepackage{setspace, graphicx, fullpage, amssymb, amsmath, epsfig, natbib, array, multirow, hyperref}
\usepackage{amsfonts, bm} 
\usepackage{dcolumn}
\usepackage{subfigure, float} 
\usepackage[margin=1in]{geometry} 
\usepackage{verbatim}
\usepackage{url}
\usepackage{enumerate}
\newcolumntype{d}[1]{D{.}{.}{#1}} 

\begin{document}

\begin{center}
\Large 22 February 2017
\end{center}

\section{Overview}

The main tasks I set out to accomplish over week as per our last meeting were as follows:

\begin{itemize}
	\item Continue showing differences between the lm/OLS and the bias-reduced logit/brglm sorting algorithms

	\item Explore the trade off of coefficients between party and ideology in the vote sorting algorithm
\end{itemize}

\noindent
Craig had previously expressed concerns with the sorting of votes on two fronts; the first being that the algorithm using the bias-reduced logistic equation for sorting votes was largely not sorting close votes as party votes and the second that it could mean very different things to have a vote sorted as a party vote when the equation has opposite signs on the coefficients for party and ideology than when these are the same. On the latter category, William hypothesized that 











\end{document}