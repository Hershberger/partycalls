\documentclass[12pt]{article}
\usepackage{setspace, graphicx, fullpage, amssymb, amsmath, epsfig, natbib, array, multirow, hyperref}
\usepackage{amsfonts, bm} 
\usepackage{dcolumn}
\usepackage{subfigure, float} 
\usepackage[margin=1.25in]{geometry} 
\usepackage{verbatim}
\usepackage{url}
\usepackage{enumerate}
\usepackage{morefloats}
\newcolumntype{d}[1]{D{.}{.}{#1}} 



\begin{document}
	
\begin{center}
	\Large 22 March 2017
\end{center}

\section{Overview}

At our previous meeting we decided to do the following:

\begin{itemize}
	\item Begin work on framing/writing an article about the role of reelection in party calls and noncalls
	
	\item Present DV/IV plots for Republicans and Democrats, the Majority and Minority caucuses, and combinations thereof
	
	\item Work with subgroups in the nonparametric analyses, especially separating cases of states with different party Senators by which is up for reelection; work to come up with explanations of why we see what we see
	
	\item Redo intra-party coefficient plots to focus on Congresses in which subgroups 
\end{itemize}

\noindent
These are included below. For analysis of reelection, William and I settled on a generalized version of difference in differences estimation, which is detailed below.

\section{First Stab at Reelection Paper}

\subsection{Introduction}

In this paper, we provide a replication of Minozzi \& Volden (2013), extending analysis into the Senate. Extending results into the Senate allows us not only to see if members respond to party pressure in the Senate as they do in the House, but also to test the role of proximity to reelection in members' behavior.

As with much work considering the behavior of members of Congress, we begin with the assumption that chief among a member's goals is reelection (Mayhew, 1974). While this would seem to indicate that members would act according to the preferences of their district above those of their party, we know from Lee (2009) that the name brands of parties confer advantages on members and that members are thus willing to take actions and positions that are either beneficial to their own party relative to the opposition. However, we also know from Carson, Koger, Lebo \& Young (2010) that following the party line too closely can be electorally costly to an individual member. It is therefore worth considering whether proximity to election changes the costs and benefits of aiding the party as perceived by members.

Levitt (1996) finds members' behavior (as proxied by ADA score) is less a factor of the party (as proxied by party leaders) and more of those of their home state (as proxied by the average ADA score of House members from their state). While these results are highly persuasive and indicative of general aspects of the decision-making process of Senators approaching reelection, what is missing is trends in member behavior relating to party influence. It remains possible that members behave differently on votes less influenced by the party in order to differentiate themselves in these years in order to still aid the party while still allowing themselves to stake out a claim of being more than a mere partisan.

Thus, a primary goals in this paper is to test how Senators roll call voting differs on votes more and less influenced by the party in Congresses which they are up for reelection compared to those which they are not. We find a decrease in member alignment to the position of the party in highly party influenced votes which is robust to different model specification not present in other votes.

%Based on previous work in the field, we expect the party to play a strong role in member vote decisions the Senate as in the House (Lee, 2009, 2015).

\subsection{Data and Methodology}

We draw on Congressional roll call data for Congresses 93-112 for both chambers in order to view the behavior of members. As in Minozzi \& Volden (2013), we iteratively sort votes based on the the predictive power of party in vote decision taken alongside ideology. We dub those votes which are significantly predicted by party as ``party calls'' and those which are not as ``noncalls.'' As in the paper we are replicating, member ideology is calculated based on the 

In order to conduct our analyses, we 

\subsection{Results}

\subsection{Conclusion}















\end{document}