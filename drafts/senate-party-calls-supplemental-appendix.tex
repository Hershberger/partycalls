\documentclass[12pt]{article}
\usepackage{setspace, graphicx, fullpage, amssymb, amsmath, epsfig, natbib, array, multirow}
\usepackage[bookmarks = false, hidelinks]{hyperref}
\usepackage{amsfonts, bm}
\usepackage{dcolumn}
\usepackage{subfigure, float}
\usepackage[margin=1in]{geometry}
\usepackage{verbatim}
\usepackage{url}
\usepackage{enumerate}
\usepackage{morefloats}
\usepackage[skip = 2pt]{caption}
\usepackage[flushleft]{threeparttable}
\usepackage[T1]{fontenc}
\usepackage{libertine}
\usepackage{scrextend}
\usepackage{setspace}
\usepackage{fancyhdr}
\usepackage{enumitem}
\usepackage{natbib}

\newcommand\fnote[1]{\captionsetup{font=normalsize}\caption*{#1}}

\setlist{nolistsep,noitemsep}
\setlength{\footnotesep}{1.2\baselineskip}
\deffootnote[.2in]{0in}{0em}{\normalsize\thefootnotemark.\enskip}
\newcolumntype{.}{D{.}{.}{-1}}
\newcolumntype{d}[1]{D{.}{.}{#1}}

\makeatother

\def\citeapos#1{\citeauthor{#1}'s (\citeyear{#1})} % possessive citation

\begin{document}

\doublespacing

%------------------------------------------------------------------------------%
% Supplemental Appendix
%------------------------------------------------------------------------------%
\setcounter{table}{0}
\setcounter{footnote}{0}
\setcounter{figure}{0}
\setcounter{page}{0}

\refstepcounter{section}
\markright{Supplemental Appendices}

\begin{center}
\vspace*{1in}
{\Large Supplemental Appendices to\\}
{\LARGE Party Calls and Reelection in the U.S. Senate}
\end{center}
\vspace*{1in}

\thispagestyle{empty}

\setcounter{tocdepth}{1}
\setcounter{secnumdepth}{0}
\tableofcontents

\renewcommand\thetable{A\arabic{table}}
\renewcommand\thepage{A\arabic{page}}

\clearpage

%------------------------------------------------------------------------------%
\subsection*{Appendix A: Classifying Party Calls}
\addcontentsline{toc}{section}{\protect\numberline{}Classifying Party Calls}%
%------------------------------------------------------------------------------%

As in \cite{Minozzi:2013}, we use an algorithm to classify votes as ``party calls''---that is, whether votes are predicted by party membership even after controlling for ideology.  We classify the complement of votes as ``party free,'' and we use them to estimate ideal points absent party influence.  The classification algorithm is iterative.  In each iteration of the algorithm, ideal points are estimated based only on the votes which were classified as ``party free'' in the previous iteration.  All votes are then regressed on ideal points and party membership, and votes are re-categorized based on the explanatory power of party in these regressions.  To begin the process we need an initial classification, and so ideal points in the first iteration are estimated using lopsided votes (which have more than 65\% or less than 35\% of members of the chamber voting on the same side).  We then use a 15 iteration ``burn-in'' period for each Congress.  During this early period, many votes switch categories from iteration to iteration.  This number of switchers also declines rapidly in these early iterations.  After burn-in, the algorithm continues until either (1) the number of votes that switch classifications stops declining, or (2) until there are fewer than 5 votes that switch.  Once either condition is met, it continues for an additional 15 iterations.  Finally, we use the last five iterations to provide a final classification of votes. During these final iterations, any vote that does not switch is classified in its appropriate category.  Any vote which does switch is dropped from further analysis, on the basis that these votes could not be credibly classified.

Our algorithm departs from MV's in a few ways, most of them minor.  However, a key change was the use of the \textsf{binIRT} command from the \textsf{R} package \textsf{emIRT} \citep{Imai:2016} to estimate ideal points, replacing the \textsf{ideal} command from the \textsf{R} package \textsf{pscl} \citep{Jackman:2015} used by MV.  The \textsf{binIRT} function is considerably faster than \textsf{ideal}, and by using it, we were able to test a much wider variety of alternative specifications to the original algorithm.

The results of these alternative specifications culminated in a few alterations to the original algorithm.  Throughout, to ensure continuity of method, we vetted alternatives by re-estimating the results from MV.  Those results are remarkably robust to the alternatives we explored.  Such robustness notwithstanding, we elected to make two minor changes.  First, the original classification algorithm used logistic regression to predict votes, meaning that party-line votes suffered from separation, which was resolved using a method for that purpose \citep{Zorn:2005}.  In this paper's classification method, we used linear models of roll call votes.  Second, the original algorithm classified votes as ``party calls'' if the $p$-value on the party indicator in a regression of a roll call vote was smaller than $0.01$.  However, the House roll call data features many more votes cast per roll call than the Senate, since the lower chamber is much larger.  As such, the threshold of $0.01$ eventuated in classifying very few Senate votes as party calls, essentially because of the smaller $n$.  We explored a variety of alternative $p$-value thresholds and settled on $0.05$, as it resulted in similar fractions of party calls across the two chambers.  Third, we also explored a variety of alternatives to the algorithm used here, but ultimately rejected them in order to maintain as much consistency with MV as possible.  These included using random initial classifications of votes, adding a ``simulated annealing''-style heating and cooling schedule to categorizations, and alternative stopping rules.  We found that none of these alternatives significantly altered the results presented in the paper, and therefore elected to use an algorithm that closely matched the early effort.

Finally, with this algorithm in hand, we probed for differences in vote classifications between the two chambers.  First, we broke down votes by close/lopsideness and classification as party calls/party free.  Table~\ref{tab-close-lop} shows these comparisons for each chamber. In each panel, there is a notable, though far from perfect, correlation between close votes and party calls.  This correlation is higher in the House ($0.51$) than in the Senate ($0.45$), but the two are remarkably close.  We take this as prima facie evidence that the classification algorithm is at work on similar data-generating processes.

\begin{table}[!htbp]
\centering
\begin{threeparttable}
\singlespacing
\caption{Party Calls and Close/Lopsided Votes}
\label{tab-close-lop}
\begin{tabular}{l cc|cc}

\hline
&\multicolumn{2}{c}{\underline{House}}&\multicolumn{2}{c}{\underline{Senate}}\\
         & Party Call      & Party Free      & Party Call      & Party Free \\
\hline
Lopsided & $4245$ $(20\%)$ & $6123$ $(29\%)$ & $2063$ $(15\%)$ & $4876$ $(35\%)$ \\
Close    & $9308$ $(45\%)$ & $1090$ $(5\%)$  & $5233$ $(37\%)$ & $1851$ $(13\%)$ \\
\hline
\end{tabular}
\begin{tablenotes}
   \item
   The threshold for a vote to be lopsided was more than 65\% of members voting on the same side of a roll call vote.
 \end{tablenotes}
\end{threeparttable}
\end{table}

Next, we focus on whether party influence exacerbates or moderates ideological tendencies.  In the regression models we use to classify votes, both ideal points and party are included as predictors of roll call behavior.  We can therefore compare the signs on the coefficients of these variables to understand how the two variables interact on the average vote.  Perhaps unsurprisingly, we find that the two coefficients have the same sign a majority of the time, regardless of chamber (see top third of Table~\ref{tab-sorting}).  Interestingly, we further find that party calls explain most of this relationship; similar signs appear for party and ideal points for about 75\% of party calls (middle of Table~\ref{tab-sorting}), yet less than 50\% of non-calls (bottom of Table~\ref{tab-sorting}).  We interpret this evidence as consistent with the idea that party calls typically act to attract extremists back into the party fold.

\begin{table}[!htbp]
\centering
\begin{threeparttable}
\singlespacing
\caption{Comparing Coefficient Signs from Roll Call Regressions}
\label{tab-sorting}
\begin{tabular}{l cc|cc}
\hline
&\multicolumn{2}{c}{\underline{House}}&\multicolumn{2}{c}{\underline{Senate}}\\
& ($-$) Ideal & ($+$) Ideal & ($-$) Ideal & ($+$) Ideal \\
\hline
All Votes \\
\hline
($-$) Party & $8159$ $(38\%)$ & $3180$ $(15\%)$ & $4581$ $(33\%)$ & $2264$ $(16\%)$ \\
($+$) Party & $3739$ $(17\%)$ & $6405$ $(30\%)$ & $3224$ $(23\%)$ & $4010$ $(28\%)$ \\
\hline
Party Calls Only\\
\hline
($-$) Party & $6522$ $(44\%)$ & $1569$ $(11\%)$ & $2812$ $(39\%)$ & $822$ $(11\%)$ \\
($+$) Party & $1627$ $(11\%)$ & $5070$ $(34\%)$ & $1045$ $(15\%)$ & $2476$ $(35\%)$ \\
\hline
Party Free Only\\
\hline
($-$) Party & $1637$ $(26\%)$ & $1611$ $(25\%)$ & $1769$ $(26\%)$ & $1442$ $(22\%)$ \\
($+$) Party & $1976$ $(31\%)$ & $1132$ $(18\%)$ & $2052$ $(31\%)$ & $1425$ $(21\%)$ \\
\hline
\end{tabular}
\begin{tablenotes}
   \item
   Each observation is a roll call vote, and the table categorizes these votes based on the signs of the Party and Ideal Point coefficients.  The Party variable is an indicator for Republican and is positively correlated with ideal points.
 \end{tablenotes}
\end{threeparttable}
\end{table}

\clearpage


%------------------------------------------------------------------------------%
\subsection*{Appendix B: Summary Statistics}
\addcontentsline{toc}{section}{\protect\numberline{}Summary Statistics}%
%------------------------------------------------------------------------------%

Here we give descriptions of report summary tables of the variables used in our paper. Members are grouped either as Democrats or Republicans, with independents being grouped with the party they caucus with in each chamber. The data are constructed with observations for members in each Congress they were present in. Values are according to member status in each Congress. Members who changed parties have multiple observations in the Congress which they did so to have observations for them in each. In each chamber, \textit{Majority} is an indicator variable for if a members' party is in the majority during a Congress, which is used to divide results and summary statistics.\footnote{\doublespacing\normalsize Each party held the majority for a portion of the $107^{\text{th}}$ Senate, with Democrats in control for most of the term.  Therefore, for the purposes of analyses, Democrats were coded as the majority of this term.  This decision does not meaningfully affect the inferences in the paper.}

The bulk of the data were provided by the Legislative Effectiveness Project \citep{Volden:2014} or constructed from those data, with a few exceptions. Keith Poole furnished the roll call data.  Committee data for all Senate terms and the $110^{\text{th}}$-$112^{\text{th}}$ in the House are from Charles Stewart's Congressional data page, with committee value ranks based on \citep{Groseclose:1998}.  Committee data from the $93^{\text{rd}}$-$109^{\text{th}}$ terms in the House come from the replication data for MV.  House elections data were provided to us by Gary Jacobson, and Senate elections data come from Dave Leip's U.S. Election Atlas.
%Senate retiree data were collected from the Congressional Bioguides.
Gingrich Senators were identified based on \citep{Theriault:2013}.

\textit{Party-Free Ideal Point} is a member's ideal point, estimated with the \textsf{binIRT} function from the \textsf{R} package \textsf{emIRT} using only party-free votes, mean-centered at zero, scaled to have unit standard deviation, and oriented so that Democrats' values are on average lower (i.e., further left) than Republicans'. \textit{Ideological Extremism} is simply the \textit{Party Free Ideal Point} value for Republicans and sign-reversed for Democrats, so that higher numbers represent more extreme members for both parties. \textit{Responsiveness} to party calls is the percentage of party calls on which a member voted with a majority of their party; \textit{Baseline Rate}  of voting with the party is that percentage for party-free votes.

\textit{Up for Reelection} is a Senate-specific variable, representing whether a member's election falls during a Congress.
%`Retiree' is an indicator variable for members who choose to leave office at or before the end of a Congress.
\textit{Vote Share} is calculated by the member's share of the vote relative to their nearest opponent.\footnote{\doublespacing\normalsize In the House, we report above or below average centered at zero since unchallenged runs were coded as missing, to avoid selecting values for these. This decision had no impact on results.}  \textit{Presidential Vote Share} in each chamber is an indication of Democrat or Republican (depending on who the member caucused with) presidential candidate 2-party vote share in the previous election based on the previous presidential election.  \textit{Party Leader} is an indicator for if a member is in one of the positions identified as the congressional leadership (other than committee positions) in the \textit{Almanac of American Politics} for a particular Congress. \textit{Committee Chair} is an indicator for whether held such a position in that Congress.  \textit{Power Committee} represents a member being on one of the top four ranked committees.  \textit{Best Committee} takes a value based on the highest ranked committee a member was on with ranks reversed so that higher means better, i.e., values range from zero (member not on a committee) to the number of committees in the chamber (member served on the highest ranked committee).  \textit{Female} is an indicator variable for female legislators.  \textit{African American} is an indicator for African American legislators.  \textit{Latino} is an indicator for Hispanic and Latino legislators.  \textit{South} is an indicator for if a member represents a state or district from 13-state south.  \textit{Seniority} is a count of consecutive terms a member has served.  \textit{Freshman} is an indicator variable for the first Congress of a member previously not in Congress.

\begin{table}[H]
\centering
\begin{threeparttable}
\singlespacing
\caption{Senate Summary Statistics}
\label{tab-senate-summary-stats}
\begin{tabular}{@{\extracolsep{5pt}}lcccc}
\\[-1.8ex]\hline
\hline \\[-1.8ex]
Variable                           & Mean   & SD     & Min     & Max \\
\hline \\[-1.8ex]
Responsiveness to Party Calls      & $85.5$  & $11.4$ & $8.8$ & $100$ \\
Party Free Ideal Point             & $0.00$  & $1.00$ & $-3.22$ & $3.40$ \\
Ideological Extremism              & $0.69$  & $0.72$ & $-1.63$ & $3.40$ \\
Baseline Rate of Voting with Party & $82.0$  & $8.2 $ & $45.1$ & $100$ \\
Up for Reelection                  & $0.33$  & $0.47$ & $0$ & $1$ \\
% Retiree                            & $0.06$  & $0.24$ & $0$ & $1$ \\
Vote Share                         & $61.2$  & $9.9 $ & $50.0$ & $100$ \\
Pres. Vote Share                   & $52.1$  & $9.7 $ & $20.1$ & $78.0$ \\
Party Leader                       & $0.10$  & $0.30$ & $0$ & $1$ \\
Committee Chair                    & $0.18$  & $0.39$ & $0$ & $1$ \\
Power Committee                    & $0.73$  & $0.45$ & $0$ & $1$ \\
Best Committee                     & $12.3$  & $2.7 $ & $0$ & $15$ \\
Female                             & $0.07$  & $0.25$ & $0$ & $1$ \\
African American                   & $<0.01$ & $0.06$ & $0$ & $1$ \\
Latino                             & $0.01$  & $0.09$ & $0$ & $1$ \\
South                              & $0.26$  & $0.44$ & $0$ & $1$ \\
Seniority                          & $6.25$  & $4.62$ & $1$ & $26$ \\
Freshman                           & $0.11$  & $0.32$ & $0$ & $1$ \\
\hline \\[-1.8ex]
\end{tabular}
\begin{tablenotes}
   \item
   $N = 1,993$
 \end{tablenotes}
\end{threeparttable}
\end{table}

\begin{table}[H]
\centering
\begin{threeparttable}
\singlespacing
\caption{House Summary Statistics}
\label{tab-house-summary-stats}
\begin{tabular}{@{\extracolsep{5pt}}lcccc}
\\[-1.8ex]\hline
\hline \\[-1.8ex]
Variable                           & Mean   & SD     & Min     & Max \\
\hline \\[-1.8ex]
Responsiveness to Party Calls      & $85.8$ & $11.5$ & $8.0$ & $100$ \\
Party Free Ideal Point             & $0.00$ & $1.00$ & $-4.08$ & $9.35$ \\
Ideological Extremism              & $0.60$ & $0.80$ & $-4.32$ & $9.35$ \\
Baseline Rate of Voting With Party & $87.0$ & $ 7.5$ & $0$     & $100$ \\
Vote Share vs. Mean Vote Share     & $ 0.0$ & $ 9.2$ & $-15.7$ & $31.4$ \\
Pres. Vote Share                   & $56.6$ & $12.4$ & $16.3$  & $96.1$ \\
Party Leader                       & $0.04$ & $0.19$ & $0$     & $1$ \\
Committee Chair                    & $0.05$ & $0.22$ & $0$     & $1$ \\
Power Committee                    & $0.26$ & $0.44$ & $0$     & $1$ \\
Best Committee                     & $13.8$ & $ 6.4$ & $0$     & $22$ \\
Female                             & $0.09$ & $0.29$ & $0$     & $1$ \\
African American                   & $0.06$ & $0.24$ & $0$     & $1$ \\
Latino                             & $0.04$ & $0.18$ & $0$     & $1$ \\
South                              & $0.30$ & $0.46$ & $0$     & $1$ \\
Seniority                          & $5.33$ & $4.05$ & $1$     & $29$ \\
Freshman                           & $0.16$ & $0.36$ & $0$     & $1$ \\
% Democrat & 0.56 & 0.50 & 0 & 1 \\
% Majority & 0.57 & 0.50 & 0 & 1 \\
\hline \\[-1.8ex]
\end{tabular}
\begin{tablenotes}
   \item
   $N = 8,544$
 \end{tablenotes}
\end{threeparttable}
\end{table}

\clearpage

%------------------------------------------------------------------------------%
\subsection*{Supplemental Appendix C: Regression Models of Responsiveness}
\addcontentsline{toc}{section}{\protect\numberline{}Regression Models of Responsiveness}%
%------------------------------------------------------------------------------%

In this appendix, we present results from regression models in the House and Senate which model \textit{Responsiveness} to party calls separately by party and majority status.  Table~\ref{tab-house-models} presents results for the House, and Table~\ref{tab-senate-models} those for the Senate.

Three sets of results are clear from these models. First, in keeping with the theory and evidence in MV, we expected that members with higher \textit{Ideological Extremism} would also have higher levels of \textit{Responsiveness}.  Indeed, even with the amendments to the classification algorithm described in Supplemental Appendix A, we see similar evidence to this effect across all subgroups in the House (first row of Table~\ref{tab-house-models}).  We see similar evidence from the Senate (first row of Table~\ref{tab-senate-models}), and, moreover, the magnitude of these coefficients is remarkably consistent across subgroup and chamber.

Second, one of the benefits of replicating MV's findings in the Senate is that there is variation in whether members were up for reelection.  We expected that reelection would make members less responsive to the call of the party as they work to pivot to their districts when approaching reelection.  The results appear in the third row of Table~\ref{tab-senate-models}.  We find first that the sign is in the expected direction and similar magnitude (about one percentage point) for all subgroups.  We also find that the coefficient achieves statistical significance for all subgroups save Democrats, for which the two-sided $p$-value rises to about XXX.

%Further, we should expect that those retiring are no longer beholden to their constituents and thus would not have this draw on their attention when the party calls. We find across all models that retirees' responsiveness to party calls takes a positive coefficient and for all, save Democrats, it is statistically significant.

We find that minority party women are substantially more responsive to party calls than their male counterparts in both chambers. Others have found that minority party women remain more focused on legislative agendas than their male counterparts \cite{Volden:2013}, and we take this finding as being in line with this account. While results for this are mixed, we find generally that increased same-party presidential vote share (an indicator of party strength in the district) increases responsiveness to party calls while increased personal vote share (an indicator of member popularity in the district) decreases responsiveness. However, this relationship does not present itself for Democrats. A number of factors complicate this relationship for Democrats, such as landslide presidential election losses and the presence of Southern Democrats early on who were more moderate than their copartisans.

\begin{table}[H]
\centering
\begin{threeparttable}
\label{tab-house-models}
\singlespacing
\small
\caption{Responsiveness to Party Calls in the U.S. House, 1973-2012}
\begin{tabular}{l c c c c c }
\hline
& All & Democrats & Republicans & Majority & Minority \\
\hline

Ideological Extremism & $7.75^{***}$ & $8.30^{***}$ & $5.88^{***}$ & $6.57^{***}$  & $8.73^{***}$ \\
                      & $(1.26)$     & $(0.89)$     & $(1.70)$     & $(1.44)$      & $(1.16)$     \\
Baseline Rate         & $0.57^{***}$ & $0.63^{***}$ & $0.41^{*}$   & $0.51^{**}$   & $0.63^{***}$ \\
                      & $(0.12)$& $(0.09)$  & $(0.19)$     & $(0.16)$      & $(0.08)$     \\
Vote Share            & $-0.01$      & $-0.05$      & $0.02$       & $-0.13^{***}$ & $-0.05$      \\
                      & $(0.03)$     & $(0.03)$     & $(0.04)$     & $(0.04)$      & $(0.04)$     \\
Pres Vote Share       & $0.03$       & $0.10$       & $-0.10$      & $0.21^{*}$    & $0.16^{*}$   \\
                      & $(0.08)$     & $(0.06)$     & $(0.10)$     & $(0.09)$      & $(0.08)$     \\
Leader                & $1.81^{**}$  & $1.96^{**}$  & $2.80^{**}$  & $2.60^{***}$  & $1.83^{*}$   \\
                      & $(0.56)$     & $(0.73)$     & $(0.99)$     & $(0.56)$      & $(0.81)$     \\
Chair                 & $4.98^{***}$ & $2.50^{**}$  & $9.72^{***}$ & $1.85^{***}$  &              \\
                      & $(0.95)$     & $(0.92)$     & $(2.02)$     & $(0.56)$      &              \\
Power Committee       & $2.76^{***}$ & $1.83^{***}$ & $2.93^{**}$  & $3.00^{***}$  & $1.07$       \\
                      & $(0.76)$     & $(0.44)$     & $(0.96)$     & $(0.87)$      & $(0.82)$     \\
Best Committee        & $-0.17$      & $-0.04$      & $-0.24$      & $-0.18$       & $-0.17$      \\
                      & $(0.10)$     & $(0.05)$     & $(0.13)$     & $(0.11)$      & $(0.11)$     \\
Female                & $1.17$       & $0.56$       & $-0.08$      & $-0.02$       & $2.18^{**}$  \\
                      & $(0.63)$     & $(0.53)$     & $(1.50)$     & $(0.66)$      & $(0.77)$     \\
African American      & $1.90$       & $-0.47$      & $5.12^{***}$ & $-2.80$       & $3.13$       \\
                      & $(1.36)$     & $(1.08)$     & $(1.31)$     & $(1.77)$      & $(1.77)$     \\
Latino                & $3.27^{**}$  & $1.78$       & $2.33$       & $2.97^{**}$   & $3.00^{*}$   \\
                      & $(1.16)$     & $(1.07)$     & $(1.71)$     & $(0.93)$      & $(1.50)$     \\
South                 & $-0.89$      & $-2.47^{**}$ & $3.56^{***}$ & $-1.77^{**}$  & $-0.54$      \\
                      & $(0.54)$     & $(0.85)$     & $(0.78)$     & $(0.59)$      & $(0.89)$     \\
Seniority             & $-0.05$      & $0.05$       & $-0.34^{*}$  & $0.03$        & $0.00$       \\
                      & $(0.06)$     & $(0.06)$     & $(0.14)$     & $(0.07)$      & $(0.10)$     \\
Freshman              & $0.78$       & $-0.07$      & $1.19$       & $0.22$        & $-0.09$      \\
                      & $(0.67)$     & $(0.53)$     & $(1.01)$     & $(0.51)$      & $(0.54)$     \\
Intercept             & $31.78^{**}$ & $25.15^{**}$ & $52.01^{*}$  & $38.88^{*}$   & $17.13^{*}$  \\
                      & $(11.77)$    & $(8.49)$     & $(20.32)$    & $(15.37)$     & $(8.60)$     \\
\hline
\end{tabular}
\begin{tablenotes}
   \item
   Results are produced by OLS regressions for all members for the entire period in the first column, with additional analyses for all Democrats and Republicans as well as all members of the Majority and Minority party in Congresses 93-112 in the House of Representatives. Details on the variables are provided in Appendix C.
   Standard errors are clustered by Congress and by member.
$^{***}p<0.001$, $^{**}p<0.01$, $^*p<0.05$
 \end{tablenotes}
\end{threeparttable}
\end{table}

\begin{table}[H]
\centering
\begin{threeparttable}
\label{tab-senate-models}
\singlespacing
\small
\caption{Responsiveness to Party Calls in the U.S. Senate, 1973-2012}
\small
\begin{tabular}{l c c c c c }
\hline
& All & Democrats & Republicans & Majority & Minority \\
\hline
Ideological Extremism & $6.23^{***}$  & $3.12^{**}$  & $7.80^{***}$   & $4.81^{***}$ & $7.99^{***}$ \\
                      & $(0.82)$      & $(1.16)$     & $(1.02)$       & $(0.99)$     & $(0.95)$     \\
Baseline Rate         & $0.74^{***}$  & $0.76^{***}$ & $0.74^{***}$   & $0.70^{***}$ & $0.72^{***}$ \\
                      & $(0.07)$      & $(0.09)$     & $(0.08)$       & $(0.07)$     & $(0.09)$     \\
Up For Reelection     & $-0.95^{***}$ & $-0.73^{**}$ & $-1.39^{**}$   & $-1.08^{**}$ & $-0.90^{*}$  \\
                      & $(0.19)$      & $(0.23)$     & $(0.48)$       & $(0.33)$     & $(0.44)$     \\
Vote Share            & $0.03$        & $-0.05^{*}$  & $0.15^{**}$    & $-0.01$      & $0.07$       \\
                      & $(0.03)$      & $(0.02)$     & $(0.05)$       & $(0.03)$     & $(0.05)$     \\
Pres Vote Share       & $0.10$        & $0.24^{***}$ & $-0.14$        & $0.18^{**}$  & $0.03$       \\
                      & $(0.05)$      & $(0.05)$     & $(0.10)$       & $(0.07)$     & $(0.12)$     \\
Leader                & $1.63^{*}$    & $2.26^{*}$   & $0.91$         & $1.56^{*}$   & $1.84$       \\
                      & $(0.72)$      & $(1.04)$     & $(0.78)$       & $(0.64)$     & $(1.09)$     \\
Chair                 & $2.10^{**}$   & $0.83$       & $3.69^{*}$     & $0.19$       &              \\
                      & $(0.79)$      & $(1.27)$     & $(1.49)$       & $(0.65)$     &              \\
Power Committee       & $-0.67$       & $-0.80$      & $-0.37$        & $-0.05$      & $-1.41$      \\
                      & $(0.72)$      & $(0.81)$     & $(1.33)$       & $(0.87)$     & $(1.19)$     \\
Best Committee        & $0.16$        & $0.22$       & $0.02$         & $0.02$       & $0.37$       \\
                      & $(0.14)$      & $(0.16)$     & $(0.25)$       & $(0.16)$     & $(0.22)$     \\
Female                & $2.03^{*}$    & $1.66^{*}$   & $0.42$         & $0.72$       & $3.48^{*}$   \\
                      & $(0.89)$      & $(0.74)$     & $(2.29)$       & $(0.65)$     & $(1.75)$     \\
African American      & $-4.69$       & $-1.00$      & $-10.99^{***}$ & $1.27$       & $-5.00^{*}$  \\
                      & $(2.46)$      & $(2.44)$     & $(1.90)$       & $(1.10)$     & $(1.95)$     \\
Latino                & $5.66^{*}$    & $1.76$       & $7.20$         & $4.68$       & $5.98$       \\
                      & $(2.52)$      & $(1.23)$     & $(4.43)$       & $(3.18)$     & $(3.44)$     \\
South                 & $0.60$        & $-1.67$      & $0.86$         & $-0.11$      & $1.21$       \\
                      & $(0.70)$      & $(0.95)$     & $(1.21)$       & $(0.72)$     & $(1.04)$     \\
Seniority             & $0.01$        & $0.05$       & $-0.02$        & $0.06$       & $0.13$       \\
                      & $(0.07)$      & $(0.11)$     & $(0.13)$       & $(0.09)$     & $(0.10)$     \\
Freshman              & $0.80$        & $0.68$       & $0.35$         & $0.38$       & $0.93$       \\
                      & $(0.56)$      & $(0.68)$     & $(0.73)$       & $(0.66)$     & $(1.06)$     \\
Intercept             & $11.90$       & $9.61$       & $18.44^{*}$    & $16.92^{*}$  & $8.99$       \\
                      & $(6.90)$      & $(7.58)$     & $(7.21)$       & $(7.32)$     & $(7.58)$     \\
\hline
R$^2$                 & 0.63          & 0.69         & 0.64           & 0.68         & 0.62         \\
Adj. R$^2$            & 0.63          & 0.68         & 0.64           & 0.67         & 0.61         \\
Num. obs.             & 1991          & 1041         & 950            & 1099         & 892          \\
RMSE                  & 6.97          & 6.12         & 7.24           & 5.91         & 7.68         \\
\hline
\end{tabular}
\begin{tablenotes}
   \item
   Results are produced by OLS regressions for all members for the entire period in the first column, with additional analyses for all Democrats and Republicans as well as all members of the Majority and Minority party in Congresses 93-112 in the Senate. Details on the variables are provided in Appendix C.
   Standard errors are clustered by Congress and by Senator.
$^{***}p<0.001$, $^{**}p<0.01$, $^*p<0.05$
 \end{tablenotes}
\end{threeparttable}
\end{table}


%------------------------------------------------------------------------------%
\subsection*{Supplemental Appendix D: Senate Reelection Fixed Effects Models}
\addcontentsline{toc}{section}{\protect\numberline{}Senate Reelection Fixed Effects Models}%
%------------------------------------------------------------------------------%

To better test the role of reelection, we use same-state senators as a natural pairing.  Table~\ref{tab-reelection} presents the results of fixed effects regression models which were summarized in Figure~2 in the main text.  The first two models include no control variables beyond the fixed effects, while the latter two also adjust for relevant control variables including lagged values of \textit{Responsiveness} to party calls, \textit{Ideological Extremism}, and \textit{Baseline Rate} of voting with the party.

\begin{table}[!htbp]
\centering
\begin{threeparttable}
\label{tab-reelection}
\singlespacing
\small
\caption{Senate Fixed Effects Models}
\begin{tabular}{l c c c c }
\hline
 & Responsiveness & Baseline Rate & Responsiveness & Baseline Rate \\
\hline
Up For Reelection                 & $-1.62^{*}$ & $-0.08$  & $-1.34^{*}$ & $0.33$       \\
                                  & $(0.63)$    & $(0.61)$ & $(0.60)$    & $(0.47)$     \\
Lag Responsiveness To Party Calls &             &          & $0.31$      & $0.06$       \\
                                  &             &          & $(0.19)$    & $(0.12)$     \\
Lag Ideological Extremism         &             &          & $4.71^{*}$  & $1.26$       \\
                                  &             &          & $(1.86)$    & $(1.08)$     \\
Lag Baseline Rate                 &             &          & $0.37^{*}$  & $0.56^{***}$ \\
                                  &             &          & $(0.16)$    & $(0.09)$     \\
Republican                        &             &          & $0.67$      & $-0.34$      \\
                                  &             &          & $(1.53)$    & $(2.02)$     \\
Majority                          &             &          & $4.49^{*}$  & $1.69$       \\
                                  &             &          & $(1.98)$    & $(2.08)$     \\
Vote Share                        &             &          & $-0.00$     & $0.00$       \\
                                  &             &          & $(0.06)$    & $(0.03)$     \\
Pres Vote Share                   &             &          & $0.01$      & $-0.04$      \\
                                  &             &          & $(0.11)$    & $(0.08)$     \\
Leader                            &             &          & $0.86$      & $0.90$       \\
                                  &             &          & $(0.64)$    & $(1.09)$     \\
Chair                             &             &          & $-0.97$     & $0.68$       \\
                                  &             &          & $(0.80)$    & $(1.21)$     \\
Power Committee                   &             &          & $0.51$      & $1.10$       \\
                                  &             &          & $(1.30)$    & $(0.99)$     \\
Best Committee                    &             &          & $-0.02$     & $-0.11$      \\
                                  &             &          & $(0.16)$    & $(0.23)$     \\
Female                            &             &          & $-0.09$     & $-0.13$      \\
                                  &             &          & $(0.58)$    & $(1.54)$     \\
African American                  &             &          & $0.66$      & $-2.79$      \\
                                  &             &          & $(2.46)$    & $(2.83)$     \\
Latino                            &             &          & $-2.89$     & $1.76$       \\
                                  &             &          & $(4.92)$    & $(6.99)$     \\
Seniority                         &             &          & $0.09$      & $0.00$       \\
                                  &             &          & $(0.13)$    & $(0.08)$     \\
\hline
Num. obs.                         & 1130        & 1130     & 952          & 952          \\
R$^2$                             & 0.71        & 0.65     & 0.93         & 0.85         \\
Adj. R$^2$                        & 0.41        & 0.30     & 0.84         & 0.66         \\
\hline
\end{tabular}
\begin{tablenotes}
   \item
   The table presents fixed effects regressions of \textit{Responsiveness} to party calls and \textit {Baseline Rate} of voting with party for the Senate, with fixed effects for Same State-Congress pairs, including 565 fixed effects for the models in the first two columns and 476 for the latter two models.  Standard errors are clustered by legislator and Congress.
   $^{***}p<0.001$, $^{**}p<0.01$, $^*p<0.05$
 \end{tablenotes}
\end{threeparttable}
\end{table}

\clearpage

%------------------------------------------------------------------------------%
\bibliographystyle{apsr}
\bibliography{senate}
%------------------------------------------------------------------------------%

\end{document}
