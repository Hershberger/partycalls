\documentclass[12pt]{article}
\usepackage{setspace, graphicx, fullpage, amssymb, amsmath, epsfig, natbib, array, multirow, hyperref}
\usepackage{amsfonts, bm} 
\usepackage{dcolumn}
\usepackage{subfigure, float} 
\usepackage[margin=1in]{geometry} 
\usepackage{verbatim}
\usepackage{url}
\usepackage{enumerate}
\newcolumntype{d}[1]{D{.}{.}{#1}} 

\begin{document}

\begin{center}
\Large 08 February 2016
\end{center}

\section{Overview}

The main tasks I set out to accomplish over week as per our last meeting were as follows:

\begin{itemize}
	\item Check that we were sorting lopsided and close votes correctly
	
		\item Figure out who is being sorted as an extremist and not as complying to party calls
	
	\item Test adding different levels of randomness to the initial lopsided/close vote seeding
	
	\item Conduct regressions with responsiveness to party calls as the dependent variable, responsiveness to noncalls and ideological extremism as the independent variables.
\end{itemize}

\noindent
After checking the first item, I worked on the others. I include the plots and tables most relevant to these below.

Over the course of this week, I continued digging into the old functions and comparing them with the new functions. I found four differences which I had previously missed. The first of these I discussed with you through email. We had moved to allowing the function to stop not only when the number of switched votes dropped below a certain threshold, but also when the number of switched votes began rising again. This is in line with the 2013 paper. Having run the algorithm with our new stopping rule, I find it to have only a negligible impact on vote sorting. Summary tables for this sorting are included.

The second through fourth, I found later in the week and thus saved for the update. None of them seem promising. The first is that we have begun dropping votes which had 4 or fewer Senators on one side. I was unable to find anything which mentioned their exclusion or inclusion in the 2013 paper or its appendices, so I looked into them. Running a single iteration of the sorting function manually initially led to more party calls than a counterpart without them, but running the algorithm all the way through proved to lead to a much larger increase in noncalls. Additionally, in the previous Senate Party Calls we were sorting votes by OLS rather than a bias-reduced logit and had party line votes dropped rather than coded as definite party calls. The first of these leads the current algorithm to be more like the one used for the 2013 paper and the second makes sense theoretically and could not possibly be driving the differences we are seeing.





\section{Tables and Figures}

\subsection{Digging into $ p < 0.05 $ Party Call Coding}

I first show plots in which characteristics I thought likely to go along with non-responsiveness are broken down by party. These are followed by tables of some particular outliers that we have. I found that the switch in responsiveness seems to happen around the ideological extremism value of 1 for both Democrats and Republicans and so I provide tables of MCs who are above this threshold and non-responsive. Finally, I conducted regression analysis with party call responsiveness as the DV and noncall response and extremism as IVs.

\begin{figure}[h]
	\caption{Main DV and Ideological Extremism - Majority and Minority Party Democrats \textit{Note}: The light gray line and dots are for Congress 107.}
	\includegraphics[width = \textwidth]{C:/Users/Ethan/Documents/GitHub/partycalls/plots/senate_p_05_dem_iv-dv_2_majority.pdf}
\end{figure}

\begin{figure}[h]
	\caption{Main DV and Ideological Extremism - Southern and Other Democrats}
	\includegraphics[width = \textwidth]{C:/Users/Ethan/Documents/GitHub/partycalls/plots/senate_p_05_dem_iv-dv_2_south.pdf}
\end{figure}

\begin{figure}[h]
	\caption{Main DV and Ideological Extremism - Majority and Minority Party Republicans \textit{Note}: The light gray line and dots are for Congress 107.}
	\includegraphics[width = \textwidth]{C:/Users/Ethan/Documents/GitHub/partycalls/plots/senate_p_05_rep_iv-dv_2_majority.pdf}
\end{figure}

\begin{figure}[h]
	\caption{Main DV and Ideological Extremism - Southern and Other Republicans}
	\includegraphics[width = \textwidth]{C:/Users/Ethan/Documents/GitHub/partycalls/plots/senate_p_05_rep_iv-dv_2_south.pdf}
\end{figure}

\begin{figure}[h]
	\caption{Main DV and Ideological Extremism - Gingrich Senators and Other Republicans}
	\includegraphics[width = \textwidth]{C:/Users/Ethan/Documents/GitHub/partycalls/plots/senate_p_05_rep_iv-dv_2_gingrich.pdf}
 \end{figure}

% latex table generated in R 3.3.2 by xtable 1.8-2 package
% Sun Feb 05 15:06:25 2017
\begin{table}[ht]
	\centering
		\caption{Democrats With Extremism $ \textgreater $ 1 and Party Call Response $ \textless $ 75\%}
	\begin{tabular}{rlrrrrr}
		\hline
		congress & mc & votes & pres vote & extremism & pfrate100 & pirate100 \\ 
		\hline
		95 & ABOUREZK (D SD) & 642 & 0.49 & 1.14 & 70.56 & 70.55 \\ 
		96 & LEAHY (D VT) & 972 & 0.44 & 1.00 & 86.56 & 74.76 \\ 
		97 & RANDOLPH (D WV) & 964 & 0.52 & 1.02 & 85.06 & 72.58 \\ 
		97 & WILLIAMS (D NJ) & 443 & 0.43 & 1.20 & 84.57 & 71.67 \\ 
		97 & KENNEDY (D MA) & 880 & 0.50 & 2.04 & 82.68 & 71.12 \\ 
		97 & PELL (D RI) & 951 & 0.56 & 1.12 & 77.89 & 72.11 \\ 
		97 & CRANSTON (D CA) & 873 & 0.41 & 1.49 & 79.21 & 74.81 \\ 
		97 & METZENBAUM (D OH) & 883 & 0.44 & 1.74 & 85.37 & 73.96 \\ 
		97 & TSONGAS (D MA) & 865 & 0.50 & 1.39 & 78.63 & 70.44 \\ 
		97 & BRADLEY (D NJ) & 917 & 0.43 & 1.25 & 82.76 & 74.65 \\ 
		101 & BIDEN (D DE) & 626 & 0.44 & 1.03 & 79.17 & 73.29 \\ 
		101 & BRADLEY (D NJ) & 625 & 0.43 & 1.19 & 75.73 & 73.46 \\ 
		\hline
	\end{tabular}
\end{table}

% latex table generated in R 3.3.2 by xtable 1.8-2 package
% Sun Feb 05 15:09:49 2017
\begin{table}[ht]
	\centering
	\caption{Republicans With Extremism $ \textgreater $ 1 and Party Call Response $ \textless $ 75\%}
	\begin{tabular}{rlrrrrr}
		\hline
		congress & mc & votes cast & pres vote & extremism & pfrate100 & pirate100 \\ 
		\hline
		93 & GOLDWATER (R AZ) & 788 & 0.67 & 1.81 & 73.70 & 72.83 \\ 
		93 & MCCLURE (R ID) & 967 & 0.71 & 1.38 & 76.39 & 72.28 \\ 
		93 & SCOTT (R VA) & 1000 & 0.69 & 1.55 & 69.64 & 67.52 \\ 
		93 & HELMS (R NC) & 1083 & 0.71 & 2.10 & 72.10 & 72.42 \\ 
		94 & GOLDWATER (R AZ) & 804 & 0.67 & 2.09 & 69.56 & 73.64 \\ 
		94 & SCOTT (R VA) & 1153 & 0.69 & 1.87 & 64.01 & 74.07 \\ 
		94 & HELMS (R NC) & 1276 & 0.71 & 2.04 & 65.90 & 74.62 \\ 
		95 & YOUNG (R ND) & 935 & 0.53 & 1.00 & 85.69 & 74.55 \\ 
		98 & SYMMS (R ID) & 615 & 0.73 & 2.38 & 70.79 & 72.95 \\ 
		98 & HELMS (R NC) & 648 & 0.51 & 1.96 & 70.49 & 67.76 \\ 
		98 & EAST (R NC) & 624 & 0.51 & 2.27 & 69.13 & 70.73 \\ 
		98 & NICKLES (R OK) & 646 & 0.63 & 1.54 & 74.26 & 66.36 \\ 
		99 & NICKLES (R OK) & 739 & 0.69 & 1.02 & 83.08 & 70.40 \\ 
		100 & HELMS (R NC) & 737 & 0.62 & 2.50 & 68.29 & 74.77 \\ 
		101 & HELMS (R NC) & 633 & 0.58 & 2.07 & 74.92 & 72.22 \\ 
		101 & HUMPHREY (R NH) & 615 & 0.63 & 1.44 & 75.74 & 71.88 \\ 
		110 & BUNNING (R KY) & 636 & 0.60 & 1.37 & 85.98 & 73.85 \\ 
		110 & KYL (R AZ) & 644 & 0.55 & 1.46 & 83.99 & 70.15 \\ 
		110 & COBURN (R OK) & 607 & 0.66 & 1.90 & 81.08 & 65.00 \\ 
		110 & VITTER (R LA) & 628 & 0.57 & 1.34 & 87.32 & 74.63 \\ 
		110 & DEMINT (R SC) & 634 & 0.59 & 2.12 & 82.44 & 66.67 \\ 
		110 & SESSIONS (R AL) & 643 & 0.63 & 1.15 & 88.37 & 70.15 \\ 
		110 & ENZI (R WY) & 636 & 0.70 & 1.39 & 87.62 & 72.46 \\ 
		112 & DEMINT (R SC) & 431 & 0.55 & 2.19 & 77.96 & 60.00 \\ 
		112 & PAUL (R KY) & 462 & 0.58 & 1.54 & 74.46 & 57.14 \\ 
		112 & LEE (R UT) & 472 & 0.65 & 1.84 & 76.15 & 59.09 \\ 
		\hline
	\end{tabular}
\end{table}

\begin{table}
	\begin{center}
		\caption{Main DV and IV Regressions}
		\begin{tabular}{l c c c c }
			\hline
			& Democrats & Republicans & Majority & Minority \\
			\hline
			(Intercept)            & $7.695^{***}$ & $10.332^{***}$ & $6.442^{**}$  & $15.341^{***}$ \\
			& $(2.098)$     & $(1.955)$      & $(2.026)$     & $(2.192)$      \\
			pfrate100              & $0.879^{***}$ & $0.847^{***}$  & $0.895^{***}$ & $0.777^{***}$  \\
			& $(0.028)$     & $(0.026)$      & $(0.026)$     & $(0.030)$      \\
			ideological\_extremism & $0.810$       & $2.350^{***}$  & $1.442^{**}$  & $2.471^{***}$  \\
			& $(0.527)$     & $(0.444)$      & $(0.481)$     & $(0.529)$      \\
			\hline
			R$^2$                  & 0.666         & 0.669          & 0.694         & 0.631          \\
			Adj. R$^2$             & 0.665         & 0.668          & 0.694         & 0.630          \\
			Num. obs.              & 1039          & 951            & 1049          & 843            \\
			RMSE                   & 6.433         & 6.758          & 6.093         & 7.120          \\
			\hline
			\multicolumn{5}{l}{\scriptsize{$^{***}p<0.001$, $^{**}p<0.01$, $^*p<0.05$}}
		\end{tabular}
	\end{center}
\end{table}
















\end{document}