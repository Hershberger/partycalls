\documentclass[12pt]{article}
\usepackage{setspace, graphicx, fullpage, amssymb, amsmath, epsfig, natbib, array, multirow}
% \usepackage{scrextend}
\usepackage[bookmarks = false, hidelinks]{hyperref}
\usepackage{amsfonts, bm}
\usepackage{dcolumn}
\usepackage{subfigure, float}
\usepackage[margin=1in]{geometry}
\usepackage{verbatim}
\usepackage{url}
\usepackage{enumerate}
\usepackage{morefloats}
\usepackage[skip = 2pt]{caption}

\usepackage[flushleft]{threeparttable}
\usepackage[T1]{fontenc}
\usepackage{libertine}
\usepackage{setspace}
\usepackage{fancyhdr}
\usepackage{enumitem}
\usepackage{natbib}

\newcommand\fnote[1]{\captionsetup{font=normalsize}\caption*{#1}}

\setlist{nolistsep,noitemsep}
\setlength{\footnotesep}{1.2\baselineskip}
% \deffootnote[.2in]{0in}{0em}{\normalsize\thefootnotemark.\enskip}
\newcolumntype{.}{D{.}{.}{-1}}
\newcolumntype{d}[1]{D{.}{.}{#1}}

\makeatother

\renewcommand{\headrulewidth}{0pt}
\def\citeapos#1{\citeauthor{#1}'s (\citeyear{#1})} % possessive citation
\bibpunct{(}{)}{;}{a}{}{,}

\begin{document}

\bibliographystyle{apsr_fs}
% \let\OLDthebibliography\thebibliography
% \renewcommand\thebibliography[1]{
%   \OLDthebibliography{#1}
%   \setlength{\parskip}{0pt}
%   \setlength{\itemsep}{1pt}
% }

\title{Party Calls and Reelection in the U.S. Senate\thanks{
We thank Gary Jacobson, Keith Poole, Charles Stewart,
and Alan Wiseman for generously sharing data
and Andrew Podob for helpful comments and conversations.
}
}

\author{
Ethan Hershberger\thanks{
  \small Ohio State University
}\quad
William Minozzi$^\dagger$\quad
Craig Volden\thanks{
  \small University of Virginia
}\\
}

\date{\today}

\maketitle

\thispagestyle{empty}
\setcounter{page}{0}

\begin{abstract}
\doublespacing
\noindent
Minozzi and Volden (2013) advance the idea that a substantial portion of partisan voting activity in Congress is a simple call to unity that is especially easily embraced by ideological extremists.  If correct, their findings should extend from the House to the Senate, despite Senate leaders lacking many of the sticks and carrots found in the House.  We adapt the theory and measurement of party calls to the Senate.  In so doing, we find that both the House and the Senate have relied heavily (and increasingly) on party calls over the past four decades.  In the Senate in particular, the lens of party calls opens new opportunities for scholars to explore partisan legislative behavior.  We take advantage of one such opportunity to show how electoral concerns limit Senators' responsiveness to party calls.
\end{abstract}

\clearpage

\doublespacing

\end{document}
