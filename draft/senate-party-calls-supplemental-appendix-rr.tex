\documentclass[12pt]{article}
\usepackage{setspace, graphicx, fullpage, amssymb, amsmath, epsfig, natbib, array, multirow}
\usepackage{scrextend}
\usepackage[bookmarks = false, hidelinks]{hyperref}
\usepackage{amsfonts, bm}
\usepackage{dcolumn}
\usepackage{subfigure, float}
\usepackage[margin=1in]{geometry}
\usepackage{verbatim}
\usepackage{url}
\usepackage{enumerate}
\usepackage{morefloats}
\usepackage[skip = 2pt]{caption}
\usepackage[flushleft]{threeparttable}
\usepackage[T1]{fontenc}
\usepackage{libertine}
\usepackage{setspace}
\usepackage{fancyhdr}
\usepackage{enumitem}
\usepackage{natbib}

\newcommand\fnote[1]{\captionsetup{font=normalsize}\caption*{#1}}

\setlist{nolistsep,noitemsep}
\setlength{\footnotesep}{1.2\baselineskip}
% \deffootnote[.2in]{0in}{0em}{\normalsize\thefootnotemark.\enskip}
\newcolumntype{.}{D{.}{.}{-1}}
\newcolumntype{d}[1]{D{.}{.}{#1}}

\makeatother

\def\citeapos#1{\citeauthor{#1}'s (\citeyear{#1})} % possessive citation

\begin{document}

\doublespacing

\setcounter{table}{0}
\setcounter{footnote}{0}
\setcounter{figure}{0}
\setcounter{page}{0}

\refstepcounter{section}
\markright{Supplemental Appendices}

\begin{center}
\vspace*{1in}
{\Large Supplemental Appendices to\\}
{\LARGE Party Calls and Reelection in the U.S. Senate}
\end{center}
\vspace*{1in}

\thispagestyle{empty}

\setcounter{tocdepth}{1}
\setcounter{secnumdepth}{0}
\tableofcontents

\renewcommand\thetable{A\arabic{table}}
\renewcommand\thepage{A\arabic{page}}

\clearpage

%------------------------------------------------------------------------------%
\subsection*{Appendix A: Classifying Party Calls}
\addcontentsline{toc}{section}{\protect\numberline{}Appendix A: Classifying Party Calls}%
%------------------------------------------------------------------------------%

As in \cite{Minozzi:2013}, we use an algorithm to classify votes as ``party calls''---that is, whether votes are predicted by party membership even after controlling for ideology.  We classify votes as ``party free'' if, in turn, they are \textit{not} reliably predicted by party membership, and we use those ``party free'' votes to estimate ideal points absent party influence.

The classification algorithm is iterative.  In each iteration of the algorithm, ideal points are estimated based only on the votes that were classified as ``party free'' in the previous iteration.  All votes are then regressed on ideal points and party membership, and votes are re-categorized based on the explanatory power of party in these regressions.  To begin the process we need an initial classification, and so ideal points in the first iteration are estimated using lopsided votes (which have more than 65\% or less than 35\% of members of the chamber voting on the same side).  We then use a 15 iteration ``burn-in'' period for each Congress.
During this early period, many votes switch categories from iteration to iteration.
The number of switchers declines rapidly in these early iterations.
After burn-in, the algorithm continues until either (1) the number of votes that switch classifications stops declining, or (2) there are fewer than five votes that switch.  Once either condition is met, the algorithm continues for an additional 15 iterations.  Finally, we use the last five iterations to provide a final classification of votes. During these final iterations, any vote that does not switch is classified in its appropriate category.  Any vote that does switch is dropped from further analysis, on the basis that these votes could not be credibly classified.

Our algorithm departs from MV's in a few ways, most of them minor.  However, a key change was the use of the \textsf{binIRT} command from the \textsf{R} package \textsf{emIRT} \citep{Imai:2016} to estimate ideal points, replacing the \textsf{ideal} command from the \textsf{R} package \textsf{pscl} \citep{Jackman:2015} used by MV.  The \textsf{binIRT} function is considerably faster than \textsf{ideal}, and by using it, we were able to test a much wider variety of alternative specifications to the original algorithm.

The results of these alternative specifications culminated in a few alterations to the original algorithm.
Throughout, to ensure continuity of method, we vetted alternatives by re-estimating the results from MV.
Those results are remarkably robust to the alternatives we explored.
Such robustness notwithstanding, we elected to make two minor changes.

The original classification algorithm used logistic regression to predict votes,
meaning that party-line votes suffered from separation, which was resolved using
a bias-reducing logit designed for that purpose \citep{Zorn:2005}.
In this paper's classification method, we used linear models of roll call votes
instead.
The classification based on linear models improved
on the bias-reduced logit-based method in three different ways.
First, classification using the linear model decreased the proportion of
unclassifiable votes from classification using the bias-reduced logit,
from $9\%$ to $3\%$ in the House and
from $2\%$ to $0.4\%$ in the Senate.
Second, party call classification based on linear models increased the positive
correlation between close votes and party calls,
from $0.18$ to $0.51$ in the House and
from $0.12$ to $0.44$ in the Senate.
Third, within each individual model of a roll call vote, the coefficients on
party and ideal point may both be positive, both negative, or have different
signs.
If party calls work to align party and ideology, then their signs should
match at higher rates than those of non-party calls.
The difference in match rates between party calls and non-calls therefore
constitutes a measure of convergent validity and therefore permits a comparison
of the two classification methods.
Again, linear models proved the better of the two.
We return to these criteria below to more fully evaluate the method used
in the paper.

The second minor change to MV's classification method was that the original algorithm classified votes as ``party calls'' if the $p$-value on the party indicator in a regression of a roll call vote was smaller than $0.01$.  However, the House roll call data features many more votes cast per roll call than the Senate, since the lower chamber is much larger.  As such, the threshold of $0.01$ eventuated in classifying few
Senate votes as party calls, essentially because of the smaller $n$.
At the $0.01$ threshold, about $15\%$ of classifiable votes were coded as party calls.
For reference, in the House, the $0.01$ threshold coded about $65\%$ of classifiable votes as party calls.
We explored a variety of alternative $p$-value thresholds and settled on $0.05$ for the Senate (keeping $p = 0.01$ for the House), as it
increased the fraction of classifiable votes coded as party calls to $52\%$.

We also explored a variety of alternatives to the algorithm used here, but ultimately rejected them in order to maintain as much consistency with MV as possible.  These included using random initial classifications of votes, adding a ``simulated annealing''-style heating and cooling schedule, and alternative stopping rules.  We found that none of these alternatives significantly altered the results presented in the paper, nor did they improve convergent validity, and therefore we elected to use an algorithm that closely matched the early effort.

Finally, with this algorithm in hand, we probed for differences in vote classifications using the criteria described briefly above.  First, we broke down votes by close/lopsideness and classification as party calls/party free.  Table~\ref{tab-close-lop} shows these comparisons for each chamber. In each panel, there is a notable, though far from perfect, correlation between close votes and party calls.  This correlation is higher in the House ($0.51$) than in the Senate ($0.44$), but the two are remarkably close.  We take this as prima facie evidence that the classification algorithm is at work on similar data-generating processes.

\begin{table}[!htbp]
\centering
\begin{threeparttable}
\singlespacing
\caption{Party Calls and Close/Lopsided Votes}
\label{tab-close-lop}
\begin{tabular}{l cc|cc}
\hline
&\multicolumn{2}{c}{\underline{House}}&\multicolumn{2}{c}{\underline{Senate}}\\
         & Party Free      & Party Call      & Party Free      & Party Call      \\
\hline
Close    & $1091$ $( 5\%)$& $9305$ $(45\%)$& $1857$ $(13\%)$& $5228$ $(37\%)$\\
Lopsided & $6122$ $(29\%)$& $4248$ $(20\%)$& $4870$ $(35\%)$& $2068$ $(15\%)$\\
\hline
\end{tabular}
\begin{tablenotes}
   \item
   The threshold for a vote to be lopsided was more than 65\% of members voting on the same side of a roll call vote.
 \end{tablenotes}
\end{threeparttable}
\end{table}

Next, we focus on whether party influence exacerbates or moderates ideological tendencies.  In the regression models we use to classify votes, both ideal points and party are included as predictors of roll call behavior.  We can therefore compare the signs on the coefficients of these variables to understand how the two variables interact on the average vote.  Perhaps unsurprisingly, we find that the two coefficients have the same sign a majority of the time, regardless of chamber (see top third of Table~\ref{tab-sorting}).  Interestingly, we further find that party calls explain most of this relationship; similar signs appear for party and ideal points for about 75\% of party calls (middle of Table~\ref{tab-sorting}), yet less than 50\% of non-calls (bottom of Table~\ref{tab-sorting}).
We interpret this evidence as consistent with the model of party calls advanced in \cite{Minozzi:2013}.

\begin{table}[!htbp]
\centering
\begin{threeparttable}
\singlespacing
\caption{Comparing Coefficient Signs from Roll Call Regressions}
\label{tab-sorting}
\begin{tabular}{l cc|cc}
% house half of table comes from analyze_vote_coding_house_lm.R
% senate half of table comes from analyze_vote_coding_senate_lm.R
\hline
&\multicolumn{2}{c}{\underline{House}}&\multicolumn{2}{c}{\underline{Senate}}\\
& ($-$) Ideal & ($+$) Ideal & ($-$) Ideal & ($+$) Ideal \\
\hline
All Votes \\
\hline
($-$) Party & $8127$ $(39\%)$& $2888$ $(14\%)$& $4581$ $(33\%)$& $2244$ $(16\%)$\\
($+$) Party & $3394$ $(16\%)$& $6357$ $(31\%)$& $3188$ $(23\%)$& $4010$ $(29\%)$\\
\hline
Party Calls Only\\
\hline
($-$) Party & $6166$ $(45\%)$& $1042$ $( 8\%)$& $2807$ $(38\%)$& $793$ $(11\%)$\\
($+$) Party & $1312$ $(10\%)$& $5033$ $(37\%)$& $1129$ $(15\%)$& $2567$ $(35\%)$\\
\hline
Party Free Only\\
\hline
($-$) Party & $1961$ $(27\%)$& $1846$ $(26\%)$& $1774$ $(26\%)$& $1451$ $(22\%)$\\
($+$) Party & $2082$ $(29\%)$& $1324$ $(18\%)$& $2059$ $(31\%)$& $1443$ $(21\%)$\\
\hline
\end{tabular}
\begin{tablenotes}
   \item
   Each observation is a roll call vote, and the table categorizes these votes based on the signs of the Party and Ideal Point coefficients
   in the vote-specific regressions that classify votes as party calls.
   The Party variable is an indicator for Republican and is positively correlated with ideal points.
 \end{tablenotes}
\end{threeparttable}
\end{table}

\clearpage


%------------------------------------------------------------------------------%
\subsection*{Appendix B: Summary Statistics}
\addcontentsline{toc}{section}{\protect\numberline{}Appendix B: Summary Statistics}%
%------------------------------------------------------------------------------%

Here we give descriptions and report summary tables of the variables used in our paper. Members are grouped either as Democrats or Republicans, with independents being grouped with the party they caucus with in each chamber. The data are constructed with observations for members in each Congress they were present in. Values are according to member status in each Congress.
Members who switched parties have one observation per party membership.
In each chamber, \textit{Majority} is an indicator variable for if a member's party is in the majority during a Congress, which is used to divide results and summary statistics.\footnote{\doublespacing\normalsize Each party held the majority for a portion of the $107^{\text{th}}$ Senate, with Democrats in control for most of the term.  Therefore, for the purposes of analyses, Democrats were coded as the majority of this term.  This decision does not meaningfully affect the inferences in the paper.}

The bulk of the data were provided by the Legislative Effectiveness Project \citep{Volden:2014} or constructed from those data, with a few exceptions. Keith Poole furnished the roll call data.  Committee data for all Senate terms and the $110^{\text{th}}$-$112^{\text{th}}$ in the House are from Charles Stewart's Congressional data page, with committee value ranks based on \cite{Groseclose:1998}.
Committee data from the $93^{\text{rd}}$-$109^{\text{th}}$ House come from the replication data for MV.  House elections data were provided to us by Gary Jacobson, and Senate elections data come from Dave Leip's U.S. Election Atlas.
%Senate retiree data were collected from the Congressional Bioguides.
Gingrich Senators were identified based on \cite{Theriault:2013}.

\textit{Party-Free Ideal Point} is a member's ideal point, estimated with the \textsf{binIRT} function from the \textsf{R} package \textsf{emIRT} using only party-free votes, mean-centered at zero, scaled to have unit standard deviation, and oriented so that Democrats' values are on average lower (i.e., further left) than Republicans'. \textit{Ideological Extremism} is simply the \textit{Party Free Ideal Point} value for Republicans and sign-reversed for Democrats, so that higher numbers represent more extreme members for both parties. \textit{Responsiveness} to party calls is the percentage of party calls on which a member voted with a majority of her party; \textit{Baseline Rate}  of voting with the party is that percentage for party-free votes.

\textit{Up for Reelection} is a Senate-specific variable, representing whether a member's election falls during a Congress.
%`Retiree' is an indicator variable for members who choose to leave office at or before the end of a Congress.
\textit{Vote Share} is calculated by the member's share of the vote relative to their nearest opponent.\footnote{\doublespacing\normalsize In the House, we report above or below average centered at zero since unchallenged runs were coded as missing, to avoid selecting values for these. This decision had no impact on results.}  \textit{Presidential Vote Share} in each chamber is an indication of Democrat or Republican (depending on who the member caucused with) presidential candidate 2-party vote share in the previous election based on the previous presidential election.  \textit{Party Leader} is an indicator for if a member is in one of the positions identified as the congressional leadership (other than committee positions) in the \textit{Almanac of American Politics} for a particular Congress. \textit{Committee Chair} is an indicator for whether the member held such a position in that Congress.  \textit{Power Committee} represents a member being on one of the top four ranked committees.  \textit{Best Committee} takes a value based on the highest ranked committee a member was on with ranks reversed so that higher means better, i.e., values range from zero (member not on a committee) to the number of committees in the chamber (member served on the highest ranked committee).  \textit{Female} is an indicator variable for female legislators.  \textit{African American} is an indicator for African American legislators.  \textit{Latino} is an indicator for Hispanic and Latino legislators.  \textit{South} is an indicator for if a member represents a state or district from 13-state South.  \textit{Seniority} is a count of consecutive terms a member has served.  \textit{Freshman} is an indicator variable for the first Congress of a member previously not in that chamber in Congress.

\begin{table}[H]
\centering
\begin{threeparttable}
\singlespacing
\caption{Senate Summary Statistics}
\label{tab-senate-summary-stats}
\begin{tabular}{@{\extracolsep{5pt}}lcccc}
\\[-1.8ex]\hline
\hline \\[-1.8ex]
Variable                           & Mean   & SD     & Min     & Max \\
\hline \\[-1.8ex]
Responsiveness                     & $85.5$  & $11.4$ & $8.8$ & $100$ \\
Party Free Ideal Point             & $0.00$  & $1.00$ & $-3.21$ & $3.38$ \\
Ideological Extremism              & $0.69$  & $0.72$ & $-1.62$ & $3.38$ \\
Baseline Rate                      & $82.0$  & $8.2 $ & $45.1$ & $100$ \\
Up for Reelection                  & $0.29$  & $0.45$ & $0$ & $1$ \\
Vote Share                         & $61.2$  & $9.9 $ & $50.0$ & $100$ \\
Pres. Vote Share                   & $52.1$  & $9.7 $ & $20.1$ & $78.0$ \\
Party Leader                       & $0.10$  & $0.30$ & $0$ & $1$ \\
Committee Chair                    & $0.18$  & $0.39$ & $0$ & $1$ \\
Power Committee                    & $0.73$  & $0.45$ & $0$ & $1$ \\
Best Committee                     & $12.3$  & $2.7 $ & $0$ & $15$ \\
Female                             & $0.07$  & $0.25$ & $0$ & $1$ \\
African American                   & $<0.01$ & $0.06$ & $0$ & $1$ \\
Latino                             & $0.01$  & $0.09$ & $0$ & $1$ \\
South                              & $0.26$  & $0.44$ & $0$ & $1$ \\
Seniority                          & $6.25$  & $4.62$ & $1$ & $26$ \\
Freshman                           & $0.11$  & $0.32$ & $0$ & $1$ \\
\hline \\[-1.8ex]
\end{tabular}
\begin{tablenotes}
   \item
   Num. Obs. $ = 1,991$.
 \end{tablenotes}
\end{threeparttable}
\end{table}

\begin{table}[H]
\centering
\begin{threeparttable}
\singlespacing
\caption{House Summary Statistics}
\label{tab-house-summary-stats}
\begin{tabular}{@{\extracolsep{5pt}}lcccc}
\\[-1.8ex]\hline
\hline \\[-1.8ex]
Variable                           & Mean   & SD     & Min     & Max \\
\hline \\[-1.8ex]
Responsiveness                     & $85.8$ & $11.5$ & $8.0$ & $100$ \\
Party Free Ideal Point             & $0.00$ & $1.00$ & $-4.05$ & $9.35$ \\
Ideological Extremism              & $0.60$ & $0.80$ & $-4.31$ & $9.35$ \\
Baseline Rate                      & $87.0$ & $ 7.5$ & $0$     & $100$ \\
Vote Share                         & $67.4$ & $11.8$ & $50$    & $100$\\
Pres. Vote Share                   & $56.6$ & $12.4$ & $16.3$  & $96.1$ \\
Party Leader                       & $0.04$ & $0.19$ & $0$     & $1$ \\
Committee Chair                    & $0.05$ & $0.22$ & $0$     & $1$ \\
Power Committee                    & $0.25$ & $0.44$ & $0$     & $1$ \\
Best Committee                     & $13.8$ & $ 6.4$ & $0$     & $22$ \\
Female                             & $0.09$ & $0.29$ & $0$     & $1$ \\
African American                   & $0.06$ & $0.24$ & $0$     & $1$ \\
Latino                             & $0.03$ & $0.18$ & $0$     & $1$ \\
South                              & $0.30$ & $0.46$ & $0$     & $1$ \\
Seniority                          & $5.33$ & $4.05$ & $1$     & $29$ \\
Freshman                           & $0.16$ & $0.36$ & $0$     & $1$ \\
\hline \\[-1.8ex]
\end{tabular}
\begin{tablenotes}
   \item
   Num. Obs. $= 8,540$.
 \end{tablenotes}
\end{threeparttable}
\end{table}

\clearpage

%------------------------------------------------------------------------------%
\subsection*{Appendix C: Regression Models of Responsiveness}
\addcontentsline{toc}{section}{\protect\numberline{}Appendix C: Regression Models of Responsiveness}%
%------------------------------------------------------------------------------%

In this appendix, we present results from regression models in the House and Senate, which model \textit{Responsiveness} to party calls separately by party and majority status.
Table~\ref{tab-house-models} presents results for the House, and Table~\ref{tab-senate-models} those for the Senate.

Two sets of results are clear from these models. First, in keeping with the theory and evidence in MV, we expected that members with higher \textit{Ideological Extremism} would also have higher levels of \textit{Responsiveness}.  Indeed, even with the amendments to the classification algorithm described in Supplemental Appendix A, we see similar evidence to this effect across all subgroups in the House (first row of Table~\ref{tab-house-models}).  We see similar evidence from the Senate (first row of Table~\ref{tab-senate-models}), and, moreover, the magnitude of these coefficients is largely consistent across subgroup and chamber.

Second, one of the benefits of replicating MV's findings in the Senate is that there is variation in whether members were up for reelection.  We expected that reelection would make members less responsive to the call of the party as they work to pivot to their districts when approaching reelection.  The results appear in the third row of Table~\ref{tab-senate-models}.  We find first that the sign is in the expected direction and similar magnitude (about one percentage point) for all subgroups.  We also find that the coefficient achieves statistical significance for all subgroups.


\begin{table}[H]
\centering
\begin{threeparttable}
\singlespacing
\small
\caption{Responsiveness to Party Calls in the U.S. House, 1973-2012}
\label{tab-house-models}
\begin{tabular}{l c c c c c }
\hline
& All & Democrats & Republicans & Majority & Minority \\
\hline
Ideological Extremism & $7.75^{***}$ & $8.30^{***}$ & $5.87^{***}$ & $6.56^{***}$  & $8.73^{***}$ \\
                      & $(1.26)$     & $(0.89)$     & $(1.70)$     & $(1.44)$      & $(1.17)$     \\
Baseline Rate         & $0.57^{***}$ & $0.63^{***}$ & $0.41^{*}$   & $0.51^{**}$   & $0.63^{***}$ \\
                      & $(0.12)$     & $(0.09)$     & $(0.19)$     & $(0.16)$      & $(0.08)$     \\
Vote Share            & $-0.01$      & $-0.05$      & $0.02$       & $-0.13^{***}$ & $-0.05$      \\
                      & $(0.03)$     & $(0.03)$     & $(0.04)$     & $(0.04)$      & $(0.04)$     \\
Pres Vote Share       & $0.03$       & $0.10$       & $-0.10$      & $0.21^{*}$    & $0.16^{*}$   \\
                      & $(0.08)$     & $(0.06)$     & $(0.10)$     & $(0.09)$      & $(0.08)$     \\
Party Leader          & $1.80^{**}$  & $1.96^{**}$  & $2.75^{**}$  & $2.60^{***}$  & $1.80^{*}$   \\
                      & $(0.56)$     & $(0.73)$     & $(0.99)$     & $(0.56)$      & $(0.81)$     \\
Committee Chair       & $4.98^{***}$ & $2.49^{**}$  & $9.72^{***}$ & $1.85^{***}$  &              \\
                      & $(0.95)$     & $(0.92)$     & $(2.02)$     & $(0.56)$      &              \\
Power Committee       & $2.76^{***}$ & $1.83^{***}$ & $2.93^{**}$  & $3.00^{***}$  & $1.07$       \\
                      & $(0.76)$     & $(0.44)$     & $(0.96)$     & $(0.87)$      & $(0.82)$     \\
Best Committee        & $-0.17$      & $-0.04$      & $-0.24$      & $-0.18$       & $-0.17$      \\
                      & $(0.10)$     & $(0.05)$     & $(0.13)$     & $(0.11)$      & $(0.11)$     \\
Female                & $1.17$       & $0.56$       & $-0.08$      & $-0.02$       & $2.17^{**}$  \\
                      & $(0.63)$     & $(0.53)$     & $(1.50)$     & $(0.66)$      & $(0.77)$     \\
African American      & $1.89$       & $-0.47$      & $5.11^{***}$ & $-2.81$       & $3.13$       \\
                      & $(1.37)$     & $(1.08)$     & $(1.31)$     & $(1.77)$      & $(1.77)$     \\
Latino                & $3.16^{**}$  & $1.76$       & $1.65$       & $2.90^{**}$   & $2.93$       \\
                      & $(1.19)$     & $(1.08)$     & $(1.60)$     & $(0.94)$      & $(1.52)$     \\
South                 & $-0.89$      & $-2.47^{**}$ & $3.58^{***}$ & $-1.77^{**}$  & $-0.53$      \\
                      & $(0.54)$     & $(0.85)$     & $(0.77)$     & $(0.59)$      & $(0.89)$     \\
Seniority             & $-0.05$      & $0.05$       & $-0.34^{*}$  & $0.03$        & $0.01$       \\
                      & $(0.06)$     & $(0.06)$     & $(0.14)$     & $(0.07)$      & $(0.10)$     \\
Freshman              & $0.79$       & $-0.07$      & $1.20$       & $0.22$        & $-0.08$      \\
                      & $(0.67)$     & $(0.53)$     & $(1.01)$     & $(0.51)$      & $(0.55)$     \\
Intercept             & $31.74^{**}$ & $25.13^{**}$ & $52.00^{*}$  & $38.83^{*}$   & $17.11^{*}$  \\
                      & $(11.79)$    & $(8.49)$     & $(20.35)$    & $(15.38)$     & $(8.60)$     \\
\hline
R$^2$                 & 0.46         & 0.63         & 0.30         & 0.57          & 0.47         \\
Adj. R$^2$            & 0.46         & 0.63         & 0.30         & 0.57          & 0.47         \\
Num. obs.             & 8540         & 4743         & 3797         & 4897          & 3643         \\
RMSE                  & 8.44         & 7.36         & 8.87         & 7.50          & 8.04         \\
\hline
\end{tabular}
\begin{tablenotes}
   \item
   Results are produced by OLS regressions for all members for the entire period in the first column, with additional analyses for all Democrats and Republicans as well as all members of the Majority and Minority party in Congresses 93-112 in the House of Representatives. Details on the variables are provided in Appendix B.
   Standard errors are clustered by Congress and by member.
$^{***}p<0.001$, $^{**}p<0.01$, $^*p<0.05$
 \end{tablenotes}
\end{threeparttable}
\end{table}

\begin{table}[H]
\centering
\begin{threeparttable}
\singlespacing
\small
\caption{Responsiveness to Party Calls in the U.S. Senate, 1973-2012}
\label{tab-senate-models}
\small
\begin{tabular}{l c c c c c }
\hline
& All & Democrats & Republicans & Majority & Minority \\
\hline
Ideological Extremism & $6.23^{***}$  & $3.12^{**}$  & $7.80^{***}$   & $4.80^{***}$ & $8.00^{***}$ \\
                      & $(0.83)$      & $(1.17)$     & $(1.02)$       & $(0.99)$     & $(0.95)$     \\
Baseline Rate         & $0.74^{***}$  & $0.76^{***}$ & $0.74^{***}$   & $0.70^{***}$ & $0.72^{***}$ \\
                      & $(0.07)$      & $(0.09)$     & $(0.08)$       & $(0.07)$     & $(0.09)$     \\
Up For Reelection     & $-0.95^{***}$ & $-0.73^{**}$ & $-1.39^{**}$   & $-1.08^{**}$ & $-0.90^{*}$  \\
                      & $(0.19)$      & $(0.23)$     & $(0.48)$       & $(0.33)$     & $(0.44)$     \\
Vote Share            & $0.03$        & $-0.06^{*}$  & $0.15^{**}$    & $-0.01$      & $0.07$       \\
                      & $(0.03)$      & $(0.02)$     & $(0.05)$       & $(0.03)$     & $(0.05)$     \\
Pres. Vote Share      & $0.10$        & $0.24^{***}$ & $-0.14$        & $0.18^{**}$  & $0.03$       \\
                      & $(0.05)$      & $(0.05)$     & $(0.10)$       & $(0.07)$     & $(0.12)$     \\
Party Leader          & $1.63^{*}$    & $2.27^{*}$   & $0.91$         & $1.56^{*}$   & $1.84$       \\
                      & $(0.72)$      & $(1.04)$     & $(0.78)$       & $(0.64)$     & $(1.08)$     \\
Committee Chair       & $2.10^{**}$   & $0.83$       & $3.69^{*}$     & $0.19$       &              \\
                      & $(0.79)$      & $(1.27)$     & $(1.49)$       & $(0.65)$     &              \\
Power Committee       & $-0.67$       & $-0.80$      & $-0.37$        & $-0.05$      & $-1.41$      \\
                      & $(0.72)$      & $(0.81)$     & $(1.33)$       & $(0.87)$     & $(1.19)$     \\
Best Committee        & $0.16$        & $0.22$       & $0.02$         & $0.02$       & $0.37$       \\
                      & $(0.14)$      & $(0.16)$     & $(0.25)$       & $(0.16)$     & $(0.22)$     \\
Female                & $2.03^{*}$    & $1.66^{*}$   & $0.42$         & $0.72$       & $3.48^{*}$   \\
                      & $(0.89)$      & $(0.74)$     & $(2.29)$       & $(0.65)$     & $(1.74)$     \\
African American      & $-4.69$       & $-1.00$      & $-10.99^{***}$ & $1.26$       & $-5.01^{*}$  \\
                      & $(2.46)$      & $(2.44)$     & $(1.89)$       & $(1.10)$     & $(1.95)$     \\
Latino                & $5.65^{*}$    & $1.75$       & $7.19$         & $4.67$       & $5.98$       \\
                      & $(2.52)$      & $(1.23)$     & $(4.43)$       & $(3.19)$     & $(3.44)$     \\
South                 & $0.60$        & $-1.68$      & $0.85$         & $-0.11$      & $1.21$       \\
                      & $(0.70)$      & $(0.95)$     & $(1.21)$       & $(0.72)$     & $(1.04)$     \\
Seniority             & $0.01$        & $0.05$       & $-0.02$        & $0.06$       & $0.13$       \\
                      & $(0.07)$      & $(0.11)$     & $(0.13)$       & $(0.09)$     & $(0.10)$     \\
Freshman              & $0.80$        & $0.68$       & $0.35$         & $0.38$       & $0.93$       \\
                      & $(0.56)$      & $(0.68)$     & $(0.73)$       & $(0.66)$     & $(1.06)$     \\
Intercept             & $11.89$       & $9.61$       & $18.42^{*}$    & $16.90^{*}$  & $8.99$       \\
                      & $(6.91)$      & $(7.59)$     & $(7.20)$       & $(7.32)$     & $(7.59)$     \\
\hline
R$^2$                 & 0.63          & 0.69         & 0.64           & 0.68         & 0.62         \\
Adj. R$^2$            & 0.63          & 0.68         & 0.64           & 0.67         & 0.61         \\
Num. obs.             & 1991          & 1041         & 950            & 1099         & 892          \\
RMSE                  & 6.97          & 6.12         & 7.24           & 5.91         & 7.68         \\
\hline
\end{tabular}
\begin{tablenotes}
   \item
   Results are produced by OLS regressions for all members for the entire period in the first column, with additional analyses for all Democrats and Republicans as well as all members of the Majority and Minority party in Congresses 93-112 in the Senate. Details on the variables are provided in Appendix B.
   Standard errors are clustered by Congress and by Senator.
$^{***}p<0.001$, $^{**}p<0.01$, $^*p<0.05$
 \end{tablenotes}
\end{threeparttable}
\end{table}


%------------------------------------------------------------------------------%
\subsection*{Appendix D: Senate Reelection Fixed Effects Models}
\addcontentsline{toc}{section}{\protect\numberline{}Appendix D: Senate Reelection Fixed Effects Models}%
%------------------------------------------------------------------------------%

To better test the role of reelection, we use same-state Senators as a natural pairing.
Table~\ref{tab-reelection} presents the results of fixed effects regression models that were summarized in Figure~3 in the main text.
In both the figure and the table, the only observations included are those
same-state pairs in which (only) one member is up for reelection.
The first two models include no control variables beyond the fixed effects, while the latter two also adjust for relevant control variables including lagged values of \textit{Responsiveness} to party calls, \textit{Ideological Extremism}, and \textit{Baseline Rate} of voting with the party.

Across both sets of findings in the table, we see Senators up for reelection being significantly less responsive to party calls than are those who are not up for reelection.  Moreover, on the party-free votes, there is no significant difference across same-state Senators.  Lacking a party call on such votes, those up for reelection are not placed in a difficult position of choosing between the party and their constituents.

\begin{table}[!htbp]
\centering
\begin{threeparttable}
\singlespacing
\small
\caption{Senate Fixed Effects Models}
\label{tab-reelection}
\begin{tabular}{l c c c c }
\hline
 & Responsiveness & Baseline Rate & Responsiveness & Baseline Rate \\
\hline
Up For Reelection                 & $-1.62^{*}$ & $-0.08$  & $-1.34^{*}$ & $0.33$       \\
                                  & $(0.63)$    & $(0.61)$ & $(0.60)$    & $(0.47)$     \\
Lag Responsiveness                &             &          & $0.31$      & $0.06$       \\
                                  &             &          & $(0.19)$    & $(0.12)$     \\
Lag Ideological Extremism         &             &          & $4.72^{*}$  & $1.27$       \\
                                  &             &          & $(1.86)$    & $(1.09)$     \\
Lag Baseline Rate                 &             &          & $0.37^{*}$  & $0.56^{***}$ \\
                                  &             &          & $(0.16)$    & $(0.09)$     \\
Republican                        &             &          & $0.67$      & $-0.34$      \\
                                  &             &          & $(1.53)$    & $(2.02)$     \\
Majority                          &             &          & $4.49^{*}$  & $1.69$       \\
                                  &             &          & $(1.97)$    & $(2.08)$     \\
Vote Share                        &             &          & $-0.00$     & $0.00$       \\
                                  &             &          & $(0.06)$    & $(0.03)$     \\
Pres. Vote Share                  &             &          & $0.01$      & $-0.04$      \\
  (oriented by Senator's party)   &             &          & $(0.11)$    & $(0.08)$     \\
Party Leader                      &             &          & $0.86$      & $0.90$       \\
                                  &             &          & $(0.65)$    & $(1.09)$     \\
Committee Chair                   &             &          & $-0.97$     & $0.68$       \\
                                  &             &          & $(0.80)$    & $(1.21)$     \\
Power Committee                   &             &          & $0.51$      & $1.10$       \\
                                  &             &          & $(1.30)$    & $(0.99)$     \\
Best Committee                    &             &          & $-0.02$     & $-0.11$      \\
                                  &             &          & $(0.16)$    & $(0.23)$     \\
Female                            &             &          & $-0.09$     & $-0.13$      \\
                                  &             &          & $(0.58)$    & $(1.54)$     \\
African American                  &             &          & $0.66$      & $-2.79$      \\
                                  &             &          & $(2.46)$    & $(2.83)$     \\
Latino                            &             &          & $-2.88$     & $1.76$       \\
                                  &             &          & $(4.92)$    & $(6.99)$     \\
Seniority                         &             &          & $0.09$      & $0.00$       \\
                                  &             &          & $(0.13)$    & $(0.08)$     \\
\hline
Num. obs.                         & 1130        & 1130     & 952         & 952          \\
R$^2$% (full model)
                                  & 0.71        & 0.65     & 0.93        & 0.85         \\
% R$^2$ (proj model)                & 0.02        & 0.00     & 0.72        & 0.52         \\
Adj. R$^2$% (full model)
                                  & 0.41        & 0.30     & 0.84        & 0.66         \\
% Adj. R$^2$ (proj model)           & -0.97       & -1.00    & 0.36        & -0.10        \\
\hline
\end{tabular}
\begin{tablenotes}
   \item
   The table presents fixed effects regressions of \textit{Responsiveness} to party calls and \textit {Baseline Rate} of voting with party for the Senate, with fixed effects for Same State-Congress pairs, including 565 fixed effects for the models in the first two columns and 476 for the latter two models.  Standard errors are clustered by legislator and by Congress.
   $^{***}p<0.001$, $^{**}p<0.01$, $^*p<0.05$
 \end{tablenotes}
\end{threeparttable}
\end{table}

\clearpage


%------------------------------------------------------------------------------%
\subsection*{Appendix E: Comparing Party Unity Scores Responsiveness to Party Calls}
\addcontentsline{toc}{section}{\protect\numberline{}Appendix E: Comparing Party Unity Scores Responsiveness to Party Calls}%
%------------------------------------------------------------------------------%

Responsivness to Party Calls bears some similarity to the more common
\emph{Party Unity} score.
For example, \cite{Carson:2010} define Party Unity as the fraction of the time
legislators vote with their party on the subset of ``party votes,'' on which a
majority of Democrats voted against a majority of Republicans.
In this appendix, we first compare party votes with party calls, then compare
Party Unity scores with Responsiveness to Party Calls, and finally report
models of Party Unity scores that mirror the models of Responsiveness from the
main paper.
Ultimately, we conclude that party calls and Responsiveness scores are
better measures of party influence than are Party Unity scores because the
algorithm used to identify party calls is purpose-built to separate party
influence from ideology.

We first identified party votes and calculated rates of voting with the party
on this subset of roll call votes, as well as on all votes.
There is relatively high similarity between this coding scheme and that for
party calls: the fraction of votes that were coded either as (1) both non-party
calls and non-party votes, or (2) both party calls and party votes ranged from
59\% to 91\%, with an average of 78\%.

Next, we created Party Unity scores, which we defined as the fraction of votes
cast by a legislator on these party votes in line with a majority of her party.
These scores are highly correlated with Responsiveness, with within-Congress
correlations ranging from 0.65 to 0.98 for the House, and from 0.69 to 0.97 for
the Senate.
The correlations tend to increase over time, significantly so in the Senate,
with a weaker relationship in the House, based on robust linear models.
There is also a sharp increase in volatility in these correlations, as the
smallest levels also occur in recent years.

Finally, we estimated models of party unity to parallel the models of
Responsiveness in the paper.
To substitute for Baseline Rates of Party Support, we created (for lack of
a better term) \emph{Non-Party-Vote Unity} scores, which we defined as the rate
of support for the party on votes for which the majorities of the two parties
did \textbf{not} disagree.
We then fit models analogous to those from Table 1 in the main paper,
substituting in Party Unity scores for Responsiveness as the dependent variable,
and substituting in Non-Party-Vote Unity scores for Baseline Rate.
The results appear in Table~\ref{tab-party-unity-regressions}.

\begin{table}[!htbp]
\centering
\begin{threeparttable}
\caption{Models of \textit{Party Unity Scores}, 1973-2012}
\label{tab-party-unity-regressions}
\singlespacing
\begin{tabular}{l c c c }
\hline
& House & Senate & Senate \\
\hline
Ideological Extremism & $13.11^{***}$ & $20.61^{***}$ & $20.59^{***}$ \\
                      & $(3.11)$      & $(1.72)$      & $(1.73)$      \\
Non-Party-Vote Unity  & $0.47^{***}$  & $0.68^{***}$  & $0.68^{***}$  \\
                      & $(0.12)$      & $(0.07)$      & $(0.07)$      \\
Up For Reelection     &               &               & $-0.32$       \\
                      &               &               & $(0.26)$      \\
Vote Share            & $-0.09^{**}$  & $-0.01$       & $-0.01$       \\
                      & $(0.03)$      & $(0.03)$      & $(0.03)$      \\
Pres. Vote Share       & $0.18$        & $0.08$        & $0.08$        \\
                      & $(0.10)$      & $(0.05)$      & $(0.05)$      \\
Party Leader          & $3.18^{***}$  & $2.26^{**}$   & $2.26^{**}$   \\
                      & $(0.93)$      & $(0.77)$      & $(0.77)$      \\
Committee Chair       & $4.96^{***}$  & $3.23^{***}$  & $3.23^{***}$  \\
                      & $(0.88)$      & $(0.71)$      & $(0.71)$      \\
Power Committee       & $2.77^{***}$  & $0.66$        & $0.66$        \\
                      & $(0.63)$      & $(0.77)$      & $(0.77)$      \\
Best Committee        & $-0.07$       & $-0.10$       & $-0.10$       \\
                      & $(0.09)$      & $(0.16)$      & $(0.16)$      \\
Female                & $1.15$        & $0.31$        & $0.30$        \\
                      & $(0.85)$      & $(1.03)$      & $(1.03)$      \\
African American      & $0.41$        & $-10.79$      & $-10.83$      \\
                      & $(1.12)$      & $(5.75)$      & $(5.74)$      \\
Latino                & $3.17^{*}$    & $2.95$        & $2.96$        \\
                      & $(1.25)$      & $(2.78)$      & $(2.78)$      \\
South                 & $-1.64^{*}$   & $-0.83$       & $-0.83$       \\
                      & $(0.76)$      & $(0.82)$      & $(0.82)$      \\
Seniority             & $-0.13$       & $-0.10$       & $-0.10$       \\
                      & $(0.08)$      & $(0.07)$      & $(0.07)$      \\
Freshman              & $0.20$        & $0.95$        & $0.86$        \\
                      & $(0.77)$      & $(0.77)$      & $(0.74)$      \\
Intercept             & $32.66^{**}$  & $9.21$        & $9.32$        \\
                      & $(10.75)$     & $(6.49)$      & $(6.49)$      \\
\hline
R$^2$                 & 0.50          & 0.75          & 0.75          \\
Adj. R$^2$            & 0.49          & 0.75          & 0.75          \\
Num. obs.             & 8529          & 1977          & 1977          \\
RMSE                  & 10.73         & 7.78          & 7.79          \\
\hline

\end{tabular}
\begin{tablenotes}
   \item
   The table presents linear models of \textit{Party Unity} scores,
   from the 93$^{\text{rd}}$-112$^{\text{th}}$ Congresses (1973-2012).
  Standard errors are clustered by Congress and by member.\\
   $^{***}p<0.001$, $^{**}p<0.01$, $^*p<0.05$
 \end{tablenotes}
\end{threeparttable}
\end{table}

A number of general findings emerged from  Table~\ref{tab-party-unity-regressions}.
First, the hypothesis tests presented in the paper for Ideological Extremism
yield similar inferences in these models of Party Unity scores.
More importantly, a set of emergent differences between these sets of models
suggest that responsiveness scores are preferable to Party Unity scores.
The relationship between Party Unity scores and Ideological Extremism was
about two to three times larger than the relationship between Responsiveness and
Ideological Extremism.
For example, the model of Responsiveness for the House (first column of Table 1
from the paper) has an overall coefficient of 7.75 on Ideological Extremism;
the corresponding coefficient from the model of Party Unity scores is 13.1.
These coefficients are even further inflated for the Senate: the coefficient
from the Senate models in Table 1 are about 6.3, whereas both Party Unity
analogues of the Senate models in Table~\ref{tab-party-unity-regressions}
have coefficients above 20.

\begin{table}[!htbp]
\centering
\scalebox{.95}{
\begin{threeparttable}
\singlespacing
\small
\caption{Fixed Effects Models of Party Unity Scores}
\label{tab-party-unity-reelection}
\begin{tabular}{l c c c c }
\hline
 & Party Unity & Non-Party-Vote Unity & Party Unity & Non-Party-Vote Unity \\
\hline
Up For Reelection                 & $-2.56^{***}$ & $1.19^{*}$ & $-2.30^{***}$ & $1.47^{**}$   \\
                                  & $(0.69)$      & $(0.54)$   & $(0.69)$      & $(0.45)$      \\
Lag Responsiveness To Party Calls &               &            & $0.28$        & $-0.07$       \\
                                  &               &            & $(0.20)$      & $(0.09)$      \\
Lag Ideological Extremism         &               &            & $11.85^{***}$ & $-6.34^{***}$ \\
                                  &               &            & $(2.02)$      & $(1.42)$      \\
Lag Non-Party-Vote Unity          &               &            & $0.44^{**}$   & $0.61^{***}$  \\
                                  &               &            & $(0.16)$      & $(0.09)$      \\
Republican                        &               &            & $0.91$        & $-0.02$       \\
                                  &               &            & $(1.88)$      & $(1.20)$      \\
Majority                          &               &            & $4.04$        & $2.05$        \\
                                  &               &            & $(2.21)$      & $(1.63)$      \\
Vote Share                        &               &            & $0.00$        & $-0.01$       \\
                                  &               &            & $(0.06)$      & $(0.03)$      \\
Pres. Vote Share                   &               &            & $-0.03$       & $-0.01$       \\
                                  &               &            & $(0.10)$      & $(0.06)$      \\
Party Leader                      &               &            & $1.04$        & $0.82$        \\
                                  &               &            & $(1.36)$      & $(1.23)$      \\
Committee Chair                   &               &            & $-0.59$       & $0.74$        \\
                                  &               &            & $(1.67)$      & $(1.00)$      \\
Power Committee                   &               &            & $1.21$        & $1.08$        \\
                                  &               &            & $(1.42)$      & $(1.65)$      \\
Best Committee                    &               &            & $-0.08$       & $-0.11$       \\
                                  &               &            & $(0.24)$      & $(0.31)$      \\
Female                            &               &            & $0.54$        & $0.82$        \\
                                  &               &            & $(1.77)$      & $(1.71)$      \\
African American                  &               &            & $-7.21$       & $3.50$        \\
                                  &               &            & $(9.19)$      & $(4.12)$      \\
Latino                            &               &            & $-5.47$       & $6.41$        \\
                                  &               &            & $(7.72)$      & $(7.48)$      \\
Seniority                         &               &            & $0.07$        & $0.05$        \\
                                  &               &            & $(0.15)$      & $(0.11)$      \\
\hline
Num. obs.                         & 1122          & 1120       & 946           & 944           \\
R$^2$% (full model)
                                  & 0.67          & 0.69       & 0.94          & 0.86          \\
% R$^2$ (proj model)                & 0.02          & 0.02       & 0.79          & 0.51          \\
Adj. R$^2$% (full model)
                                  & 0.34          & 0.38       & 0.85          & 0.68          \\
% Adj. R$^2$ (proj model)           & -0.98         & -0.99      & 0.52          & -0.13         \\
\hline
\end{tabular}

\begin{tablenotes}
   \item
   The table presents fixed effects regressions of \textit{Party Unity} to party calls and \textit {Non-Party-Vote Unity} scores for the Senate, with fixed effects for Same State-Congress pairs, including 561 fixed effects for the models in the first two columns and 473 for the latter two models.  Standard errors are clustered by legislator and by Congress.
   $^{***}p<0.001$, $^{**}p<0.01$, $^*p<0.05$
 \end{tablenotes}
\end{threeparttable}
}
\end{table}

Furthermore, analogous models of Party Unity scores that mimic the Senate-pair
design yield a similar finding.
Table~\ref{tab-party-unity-reelection} mimics the models in
Table~\ref{tab-reelection}.
Again, the coefficients on whether a senator is Up for Reelection for
Party Unity are twice the
magnitude of those for Responsiveness.
And the party-unity models also fail our placebo test.
The coefficients on Up for Reelection are positive and significant
($p < 0.01$).
Thus, models of Party Unity scores yield both implausibly large coefficient
magnitudes and a failed placebo test.

Based on these analyses, we draw two main conclusions.
First, Party Unity scores and Responsiveness to Party Calls both capture much of
the same ``signal.''
Second, however, Party Unity scores suffer from their underlying blunt
categorization rule, which was precisely the impetus for the algorithm
developed in \cite{Minozzi:2013}. The failed placebo test resonates strongly
with the modal scholarly interpretation of
Party Unity scores as ``contaminated'' by endogeneity and ideology,
an interpretation which led \cite{Carson:2010}, for example, to rely on an
instrumental variables strategy to examine the effects of party voting.

\clearpage

%------------------------------------------------------------------------------%
\singlespacing
\bibliographystyle{apsr_fs}
\bibliography{senate}
%------------------------------------------------------------------------------%

\end{document}
